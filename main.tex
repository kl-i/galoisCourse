\documentclass{article}
\usepackage[left=1in,right=1in]{geometry}
\usepackage{subfiles}
\usepackage{amsmath, amssymb, stmaryrd, verbatim} % math symbols
\usepackage{amsthm} % thm environment
\usepackage{mdframed} % Customizable Boxes
\usepackage{hyperref,nameref,cleveref,enumitem} % for references, hyperlinks
\usepackage[dvipsnames]{xcolor} % Fancy Colours
\usepackage{mathrsfs} % Fancy font
\usepackage{bbm} % mathbb numerals
\usepackage[bbgreekl]{mathbbol} % for bb Delta
\usepackage{tikz, tikz-cd, float} % Commutative Diagrams
\usetikzlibrary{decorations.pathmorphing} % for squiggly arrows in tikzcd
\usepackage{perpage}
\usepackage{parskip} % So that paragraphs look nice
\usepackage{ifthen,xargs} % For defining better commands
\usepackage[T1]{fontenc}
\usepackage[utf8]{inputenc}
\usepackage{tgpagella}
\usepackage{cancel}

% Bibliography
% \usepackage{url}
\usepackage[backend=biber,
            isbn=false,
            doi=false,
            giveninits=true,
            style=alphabetic,
            useprefix=true,
            maxcitenames=4,
            maxbibnames=4,
            sorting=nyt,
            citestyle=alphabetic]{biblatex}

% Shortcuts

% % Local to this project
\setcounter{secnumdepth}{4}
\renewcommand\labelitemi{--} % Makes itemize use dashes instead of bullets
\newcommand{\bbrkt}[1]{\llbracket #1 \rrbracket}
\newcommand{\IND}{\mathrm{Ind}}
\newcommand{\AFF}{\mathrm{Aff}}
\newcommand{\FSCH}{\mathrm{fSch}}
\newcommand{\SCH}{\mathrm{Sch}}
\DeclareMathOperator{\SPEC}{Spec}
\DeclareMathOperator{\QCOH}{QCoh}
\DeclareMathOperator{\INDCOH}{IndCoh}
\newcommand{\DR}{\mathrm{dR}}
\newcommand{\SYM}{\mathrm{Sym}}
\newcommand{\HOM}{\mathrm{Hom}}
\newcommand{\MAP}{\mathrm{Map}}
\newcommand{\CALG}{\mathrm{cAlg}}
\DeclareMathOperator{\GL}{GL}
\DeclareMathOperator{\LIE}{Lie}
\newcommand{\AD}{\mathrm{Ad}}
\DeclareMathOperator{\MAX}{Max}
\DeclareMathOperator{\FROB}{Frob}
\newcommand{\SEP}{\mathrm{sep}}
\newcommand{\LOC}{\mathrm{Loc}}
\DeclareMathOperator{\PERV}{Perv}
\newcommand{\RED}{\mathrm{red}}
\newcommand{\ZAR}{\mathrm{Zar}}
\newcommand{\FPPF}{\mathrm{fppf}}
\newcommand{\ET}{\text{ét}}
\DeclareMathOperator{\SPF}{Spf}
\newcommand{\CTS}{\mathrm{cts}}
\newcommand{\INF}{\mathrm{Inf}}
\DeclareMathOperator{\STK}{Stk}
\DeclareMathOperator{\PSTK}{PStk}
\DeclareMathOperator{\SSET}{sSet}
\DeclareMathOperator{\BUN}{Bun}
\DeclareMathOperator{\PSH}{PSh}
\DeclareMathOperator{\SH}{Sh}
\newcommand{\FPQC}{\mathrm{fpqc}}
\DeclareMathOperator{\SPH}{Sph}
\newcommand{\FT}{\mathrm{ft}}
\DeclareMathOperator{\CRYS}{Crys}
\DeclareMathOperator{\GR}{Gr}
\DeclareMathOperator{\REP}{Rep}
\newcommand{\HITCH}{\mathrm{Hitch}}
\newcommand{\CRIT}{\mathrm{crit}}
\DeclareMathOperator{\LOCSYS}{LocSys}
\newcommand{\GRPD}{\mathrm{Grpd}}
\DeclareMathOperator{\FUN}{Fun}
\newcommand{\HECKE}{\mathrm{Hecke}}
\newcommand{\TRIV}{\mathrm{Triv}}
\newcommand{\REGLUE}{\mathrm{ReGlue}}
\newcommand{\LVL}{\mathrm{lvl}}
\newcommand{\PT}{\mathrm{pt}}
\DeclareMathOperator{\COH}{Coh}
\DeclareMathSymbol\bbDe  \mathord{bbold}{"01} % blackboard Delta
\newcommand{\CL}{\mathrm{cl}}
\newcommand{\OPER}{\operatorname{Oper}}
\DeclareMathOperator{\PERF}{Perf}
\DeclareMathOperator{\INV}{Inv}
\newcommand{\PIC}{\mathrm{Pic}}
\newcommand{\RAT}{\mathrm{Rat}}
\newcommand{\DIV}{\mathrm{Div}}
\newcommand{\SPA}{\mathrm{Spa}\,}
\newcommand{\SPD}{\mathrm{Spd}\,}
\newcommand{\LA}{\mathrm{la}}
\newcommand{\SNORM}{\mathrm{SNorm}}
\newcommand{\NORM}{\mathrm{Norm}}
\newcommand{\BAN}{\mathrm{Ban}}
\newcommand{\CYC}{\mathrm{cyc}}


% % Misc
\newcommand{\brkt}[1]{\left(#1\right)}
\newcommand{\sqbrkt}[1]{\left[#1\right]}
\newcommand{\dash}{\text{-}}

% % Logic
\renewcommand{\implies}{\Rightarrow}
\renewcommand{\iff}{\Leftrightarrow}
\newcommand{\limplies}{\Leftarrow}
\newcommand{\NOT}{\neg\,}
\newcommand{\AND}{\, \land \,}
\newcommand{\OR}{\, \lor \,}
\newenvironment{forward}{($\implies$)}{}
\newenvironment{backward}{($\limplies$)}{}

% % Sets
\DeclareMathOperator{\supp}{supp}
\newcommand{\set}[1]{\left\{#1\right\}}
\newcommand{\st}{\text{ s.t. }}
\newcommand{\minus}{\setminus}
\newcommand{\subs}{\subseteq}
\newcommand{\ssubs}{\subsetneq}
\newcommand{\sups}{\supseteq}
\newcommand{\ssups}{\supset}
\DeclareMathOperator{\im}{Im}
\newcommand{\nothing}{\varnothing}
\DeclareMathOperator{\join}{\sqcup}
\DeclareMathOperator{\meet}{\sqcap}

% % Greek 
\newcommand{\al}{\alpha}
\newcommand{\be}{\beta}
\newcommand{\ga}{\gamma}
\newcommand{\de}{\delta}
\newcommand{\ep}{\varepsilon}
\newcommand{\ph}{\varphi}
\newcommand{\io}{\iota}
\newcommand{\ka}{\kappa}
\newcommand{\la}{\lambda}
\newcommand{\om}{\omega}
\newcommand{\si}{\sigma}

\newcommand{\Ga}{\Gamma}
\newcommand{\De}{\Delta}
\newcommand{\Th}{\Theta}
\newcommand{\La}{\Lambda}
\newcommand{\Si}{\Sigma}
\newcommand{\Om}{\Omega}

% % Mathbb
\newcommand{\bA}{\mathbb{A}}
\newcommand{\bB}{\mathbb{B}}
\newcommand{\bC}{\mathbb{C}}
\newcommand{\bD}{\mathbb{D}}
\newcommand{\bE}{\mathbb{E}}
\newcommand{\bF}{\mathbb{F}}
\newcommand{\bG}{\mathbb{G}}
\newcommand{\bH}{\mathbb{H}}
\newcommand{\bI}{\mathbb{I}}
\newcommand{\bJ}{\mathbb{J}}
\newcommand{\bK}{\mathbb{K}}
\newcommand{\bL}{\mathbb{L}}
\newcommand{\bM}{\mathbb{M}}
\newcommand{\bN}{\mathbb{N}}
\newcommand{\bO}{\mathbb{O}}
\newcommand{\bP}{\mathbb{P}}
\newcommand{\bQ}{\mathbb{Q}}
\newcommand{\bR}{\mathbb{R}}
\newcommand{\bS}{\mathbb{S}}
\newcommand{\bT}{\mathbb{T}}
\newcommand{\bU}{\mathbb{U}}
\newcommand{\bV}{\mathbb{V}}
\newcommand{\bW}{\mathbb{W}}
\newcommand{\bX}{\mathbb{X}}
\newcommand{\bY}{\mathbb{Y}}
\newcommand{\bZ}{\mathbb{Z}}
\newcommand{\id}{\mathbbm{1}}

% % Mathcal
\newcommand{\cA}{\mathcal{A}}
\newcommand{\cB}{\mathcal{B}}
\newcommand{\cC}{\mathcal{C}}
\newcommand{\cD}{\mathcal{D}}
\newcommand{\cE}{\mathcal{E}}
\newcommand{\cF}{\mathcal{F}}
\newcommand{\cG}{\mathcal{G}}
\newcommand{\cH}{\mathcal{H}}
\newcommand{\cI}{\mathcal{I}}
\newcommand{\cJ}{\mathcal{J}}
\newcommand{\cK}{\mathcal{K}}
\newcommand{\cL}{\mathcal{L}}
\newcommand{\cM}{\mathcal{M}}
\newcommand{\cN}{\mathcal{N}}
\newcommand{\cO}{\mathcal{O}}
\newcommand{\cP}{\mathcal{P}}
\newcommand{\cQ}{\mathcal{Q}}
\newcommand{\cR}{\mathcal{R}}
\newcommand{\cS}{\mathcal{S}}
\newcommand{\cT}{\mathcal{T}}
\newcommand{\cU}{\mathcal{U}}
\newcommand{\cV}{\mathcal{V}}
\newcommand{\cW}{\mathcal{W}}
\newcommand{\cX}{\mathcal{X}}
\newcommand{\cY}{\mathcal{Y}}
\newcommand{\cZ}{\mathcal{Z}}

% % Mathfrak
\newcommand{\f}[1]{\mathfrak{#1}}

% % Mathrsfs
\newcommand{\s}[1]{\mathscr{#1}}

% % Category Theory
\DeclareMathOperator{\obj}{Obj}
\DeclareMathOperator{\END}{End}
\DeclareMathOperator{\AUT}{Aut}
\newcommand{\CAT}{\mathrm{Cat}}
\newcommand{\SET}{\mathrm{Set}}
\newcommand{\TOP}{\mathrm{Top}}
\newcommand{\MON}{\mathrm{Mon}}
\newcommand{\GRP}{\mathrm{Grp}}
\newcommand{\AB}{\mathrm{Ab}}
\newcommand{\RING}{\mathrm{Ring}}
\newcommand{\CRING}{\mathrm{CRing}}
\newcommand{\MOD}{\mathrm{Mod}}
\newcommand{\VEC}{\mathrm{Vec}}
\newcommand{\ALG}{\mathrm{Alg}}
\newcommand{\ORD}{\mathrm{Ord}}
\newcommand{\POSET}{\mathbf{PoSet}}
\newcommand{\map}[2]{\yrightarrow[#2][2.5pt]{#1}[-1pt]}
\newcommand{\iso}[1][]{\cong_{#1}}
\newcommand{\OP}{\mathrm{op}}
\newcommand{\darrow}{\downarrow}
\newcommand{\LIM}{\varprojlim}
\newcommand{\COLIM}{\varinjlim}
\DeclareMathOperator{\coker}{coker}
\newcommand{\fall}[2]{\downarrow_{#2}^{#1}}
\newcommand{\lift}[2]{\uparrow_{#1}^{#2}}

% % Algebra
\newcommand{\nsub}{\trianglelefteq}
\newcommand{\inv}{{-1}}
\newcommand{\dvd}{\,|\,}
\DeclareMathOperator{\ev}{ev}

% % Analysis
\newcommand{\abs}[1]{\left\vert #1 \right\vert}
\newcommand{\norm}[1]{\left\Vert #1 \right\Vert}
\renewcommand{\bar}[1]{\overline{#1}}
\newcommand{\<}{\langle}
\renewcommand{\>}{\rangle}
\renewcommand{\hat}[1]{\widehat{#1}}
\renewcommand{\check}[1]{\widecheck{#1}}
\newcommand{\dsum}[2]{\sum_{#1}^{#2}}
\newcommand{\dprod}[2]{\prod_{#1}^{#2}}
\newcommand{\del}[2]{\frac{\partial#1}{\partial#2}}
\newcommand{\res}[2]{{% we make the whole thing an ordinary symbol
  \left.\kern-\nulldelimiterspace % automatically resize the bar with \right
  #1 % the function
  %\vphantom{\big|} % pretend it's a little taller at normal size
  \right|_{#2} % this is the delimiter
  }}

% % Galois
\DeclareMathOperator{\GAL}{Gal}
\DeclareMathOperator{\ORB}{Orb}
\DeclareMathOperator{\STAB}{Stab}
\newcommand{\emb}[3]{\mathrm{Emb}_{#1}(#2, #3)}
\newcommand{\Char}[1]{\mathrm{Char}#1}

%% code from mathabx.sty and mathabx.dcl to get some symbols from mathabx
\DeclareFontFamily{U}{mathx}{\hyphenchar\font45}
\DeclareFontShape{U}{mathx}{m}{n}{
      <5> <6> <7> <8> <9> <10>
      <10.95> <12> <14.4> <17.28> <20.74> <24.88>
      mathx10
      }{}
\DeclareSymbolFont{mathx}{U}{mathx}{m}{n}
\DeclareFontSubstitution{U}{mathx}{m}{n}
\DeclareMathAccent{\widecheck}{0}{mathx}{"71}

% Arrows with text above and below with adjustable displacement
% (Stolen from Stackexchange)
\newcommandx{\yaHelper}[2][1=\empty]{
\ifthenelse{\equal{#1}{\empty}}
  % no offset
  { \ensuremath{ \scriptstyle{ #2 } } } 
  % with offset
  { \raisebox{ #1 }[0pt][0pt]{ \ensuremath{ \scriptstyle{ #2 } } } }  
}

\newcommandx{\yrightarrow}[4][1=\empty, 2=\empty, 4=\empty, usedefault=@]{
  \ifthenelse{\equal{#2}{\empty}}
  % there's no text below
  { \xrightarrow{ \protect{ \yaHelper[ #4 ]{ #3 } } } } 
  % there's text below
  {
    \xrightarrow[ \protect{ \yaHelper[ #2 ]{ #1 } } ]
    { \protect{ \yaHelper[ #4 ]{ #3 } } } 
  } 
}

% xcolor
\definecolor{darkgrey}{gray}{0.10}
\definecolor{lightgrey}{gray}{0.30}
\definecolor{slightgrey}{gray}{0.80}
\definecolor{softblue}{RGB}{30,100,200}

% hyperref
\hypersetup{
      colorlinks = true,
      linkcolor = {softblue},
      citecolor = {blue}
}

\newcommand{\link}[1]{\hypertarget{#1}{}}
\newcommand{\linkto}[2]{\hyperlink{#1}{#2}}

% Perpage
\MakePerPage{footnote}

% Theorems

% % custom theoremstyles
\newtheoremstyle{definitionstyle}
{5pt}% above thm
{0pt}% below thm
{}% body font
{}% space to indent
{\bf}% head font
{\vspace{1mm}}% punctuation between head and body
{\newline}% space after head
{\thmname{#1}\thmnote{\,\,--\,\,#3}}

\newtheoremstyle{exercisestyle}%
{5pt}% above thm
{0pt}% below thm
{\it}% body font
{}% space to indent
{\it}% head font
{.}% punctuation between head and body
{ }% space after head
{\thmname{#1}\thmnote{ (#3)}}

\newtheoremstyle{examplestyle}%
{5pt}% above thm
{0pt}% below thm
{\it}% body font
{}% space to indent
{\it}% head font
{.}% punctuation between head and body
{\newline}% space after head
{\thmname{#1}\thmnote{ (#3)}}

\newtheoremstyle{remarkstyle}%
{5pt}% above thm
{0pt}% below thm
{}% body font
{}% space to indent
{\it}% head font
{.}% punctuation between head and body
{ }% space after head
{\thmname{#1}\thmnote{\,\,--\,\,#3}}

\newtheoremstyle{questionstyle}%
{5pt}% above thm
{0pt}% below thm
{}% body font
{}% space to indent
{\it}% head font
{?}% punctuation between head and body
{ }% space after head
{\thmname{#1}\thmnote{\,\,--\,\,#3}}

% Custom Environments

% % Theorem environments

\theoremstyle{definitionstyle}
\newmdtheoremenv[
    linewidth = 2pt,
    leftmargin = 0pt,
    rightmargin = 0pt,
    linecolor = darkgrey,
    topline = false,
    bottomline = false,
    rightline = false,
    footnoteinside = true
]{dfn}{Definition}
\newmdtheoremenv[
    linewidth = 2 pt,
    leftmargin = 0pt,
    rightmargin = 0pt,
    linecolor = darkgrey,
    topline = false,
    bottomline = false,
    rightline = false,
    footnoteinside = true
]{prop}{Proposition}
\newmdtheoremenv[
    linewidth = 2 pt,
    leftmargin = 0pt,
    rightmargin = 0pt,
    linecolor = darkgrey,
    topline = false,
    bottomline = false,
    rightline = false,
    footnoteinside = true
]{cor}{Corollary}

\theoremstyle{exercisestyle}
\newmdtheoremenv[
    linewidth = 0.7 pt,
    leftmargin = 20pt,
    rightmargin = 0pt,
    linecolor = darkgrey,
    topline = false,
    bottomline = false,
    rightline = false,
    footnoteinside = true
]{ex}{Exercise}
\newmdtheoremenv[
    linewidth = 0.7 pt,
    leftmargin = 20pt,
    rightmargin = 0pt,
    linecolor = darkgrey,
    topline = false,
    bottomline = false,
    rightline = false,
    footnoteinside = true
]{lem}{Lemma}

\theoremstyle{examplestyle}
\newmdtheoremenv[
    linewidth = 0.7 pt,
    leftmargin = 0pt,
    rightmargin = 0pt,
    linecolor = darkgrey,
    topline = false,
    bottomline = false,
    rightline = false,
    footnoteinside = true
]{eg}{Example}
\newmdtheoremenv[
    linewidth = 0.7 pt,
    leftmargin = 0pt,
    rightmargin = 0pt,
    linecolor = darkgrey,
    topline = false,
    bottomline = false,
    rightline = false,
    footnoteinside = true
]{ceg}{Counter Example}

\theoremstyle{remarkstyle}
\newtheorem{rmk}{Remark}

\theoremstyle{questionstyle}
\newtheorem{question}{Question}

\newenvironment{proof1}{
  \begin{proof}\renewcommand\qedsymbol{$\blacksquare$}
}{
  \end{proof}
} % Proofs ending with black qedsymbol 

% % tikzcd diagram 
\newenvironment{cd}{
    \begin{figure}[H]
    \centering
    \begin{tikzcd}
}{
    \end{tikzcd}
    \end{figure}
}

% tikzcd
% % Substituting symbols for arrows in tikz comm-diagrams.
\tikzset{
  symbol/.style={
    draw=none,
    every to/.append style={
      edge node={node [sloped, allow upside down, auto=false]{$#1$}}}
  }
}

\addbibresource{mybib.bib}

\begin{document}

\title{An 8 hours course in Galois theory}

\author{Ken Lee}
\date{Autumn 2024}
\maketitle

\tableofcontents

\section{The main theorem of Galois theory}

We show an example of the fundamental theorem of Galois theory.
Consider the polynomial $f(T) = T^3 - 2 \in \bQ[T]$.
Let $\al_0 , \al_1 , \al_2 \in \bC$ be the roots of $f$.
\[
  \text{Slogan : Galois theory studies the ``symmetries'' of 
  roots of polynomials}
\]
To make this precise, let us first investigate the field obtained
by chucking in $\al_0 , \al_1 , \al_2$ to $\bQ$.
Define \[
  \bQ_f := \bQ(\al_0 , \al_1 , \al_2) := 
  \text{ smallest field in $\bC$ containing $\bQ , \al_0 , \al_1 ,\al_2$}
\]
\textbf{Question 0 : What does $\bQ_f$ look like?}
We try to describe $\bQ(\al_0)$ first.
Consider the map $T \mapsto \al_0$ 
\begin{cd}
	{\mathbb{Q}[T]} & {\mathbb{C}} \\
	{\mathbb{Q}(\alpha_0)}
	\arrow["{T \mapsto \al_0}", from=1-1, to=1-2]
	\arrow[dashed, from=1-1, to=2-1]
	\arrow["\subseteq"', from=2-1, to=1-2]
\end{cd}
The image is $\bQ[\al_0]$ the collection of polynomial expressions
in $\al_0$ with coefficients in $\bQ$.
Since $f \in \bQ[T]$ is irreducible\footnote{
  Can be checked by Eisenstein's criterion.
  Alternatively, a cubic over $\bQ$ is reducible iff
  it has a root in $\bQ$.
  This can be checked to be impossible by brute force.
}
we have $\bQ[\al_0] = \bQ[T] / (f)$
and hence this has a $\bQ$-basis $1 , \al_0 , \al_0^2$.
\begin{itemize}
  \item \textit{
    Exercise 1 :  show that for a field $K$ and 
    an $K$-algebra $A$ which is finite dimensional 
    as a $K$-vector space and an integral domain,
    $A$ must be field.
  }
\end{itemize}
It follows that $\bQ[\al_0]$ is a field and hence
\[
  \bQ[\al_0] = \bQ(\al_0)
\]
Now we do a trick by observing that \[
  \brkt{\frac{\al_1}{\al_0}}^3 = 2 / 2 = 1
\]
Later on, we will give a way of checking when a polynomial has repeated roots
so assume for now that all $\al_0 , \al_1 , \al_2$ are distinct.
Then we get $\al_1 = \al_0 \om$ for some $\om \neq 1 = \om^3$,
and similarly $\al_2 = \al_0 \om^2$.
The $\om, \om^2$ here are called a \emph{primitive cube roots of unity}.
They are both roots of the polynomial $T^2 + T + 1 \in \bQ[T]$.
\begin{itemize}
  \item \textit{Exercise 2 : Show that $\om \notin \bQ(\al_0)$.\footnote{
    Hint : Suppose $\om = \la_0 + \la_1 \al_0 + \la_2 \al_0^2$
    with $\la_i \in \bQ$. Then using $\om^2 + \om + 1 = 0$
    show that $\al_0$ satisfies a degree four polynomial over $\bQ$.
    Then deduce that $\al_0 \in \bQ$ for a contradiction.
  }}
\end{itemize}
Since $T^2 + T + 1$ is degree two and does not have a root in $\bQ(\al_0)$,
it is irreducible in $\bQ(\al_0)[T]$
and so \[
  \bQ[\al_0 , \om] \simeq \bQ[\al_0][T] / (T^2 + T + 1)
\]
As a $\bQ[\al_0]$-vector space, this has dimension two and
hence is again a field by Exercise 1. 
We deduce 
\textbf{Answer 0 :} \[
  \bQ(\al_0 , \al_1 , \al_2) = \bQ[\al_0 , \om] = \bQ[\al_0 , \al_1 ,\al_2]
\]

We now define \emph{the Galois group of $f$} as \[
  G_f := \AUT_\bQ \bQ(\al_0 , \al_1 , \al_2)
  := \set{\si : \bQ_f \to \bQ_f \st \si \text{ ring morphism and }
  \forall\,\la \in \bQ\,,\,\si(\la) = \la}
\]
\textbf{Question 1 : Why is this the ``symmetries'' of $\al_0 , \al_1 ,\al_2$?}
Observation : any $\si \in G_f$ must permute $\set{\al_0 , \al_1 , \al_2}$.
This is \textbf{\emph{the} trick} that underlies Galois theory : \[
  f(\si(\al_i)) = (\si(\al_i))^3 - 2 = \si(\al_i^3 - 2) = 0
\]
Hence we have a well-define group morphism \[
  G_f \to \AUT\set{\al_0 , \al_1 , \al_2}
\]
Since $\bQ_f = \bQ[\al_0 , \al_1 ,\al_2]$ any $\si \in G_f$
is determined by what it does on $\al_i$ hence the above morphism is injective.
\textbf{Answer 1 : The above morphism defines an isomorphism}
\[
  G_f \simeq \set{
    \si \in \AUT\set{\al_0 ,\al_1 ,\al_2} \st 
    \forall g \in \bQ[X_0 , X_1 , X_2] \,,
    g(\al_0 , \al_1 ,\al_2) = 0 \implies
    g(\si(\al_0) , \si(\al_1) , \si(\al_2)) = 0
  }
\]
\textbf{in other words, $G_f$ is the permutations of roots of $f$
which preserves all algebraic relations over $\bQ$.}
\begin{proof}
  $\bQ[\al_0 , \al_1 , \al_2]$ is precisely the image of 
  the evaluation map \[
    \bQ[X_0 , X_1 , X_2] \to \bQ[\al_0 , \al_1 , \al_2] \,,\,X_i \mapsto \al_i
  \]
  It follows that $
    \bQ[\al_0 , \al_1 , \al_2] \simeq 
    \bQ[X_0 , X_1 , X_2] / I
  $ where $I$ is the set of polynomials $g(X_0 , X_1 , X_2)$ with
  $g(\al_0 , \al_1 , \al_2)$.
  From this, it is clear that $G_f$ lands inside the RHS.
  Now given $\tilde{\si}$ in RHS,
  one can evaluate \[
    \bQ[X_0 , X_1 , X_2] \to \bQ[\al_0 , \al_1 ,\al_2] \,,\, 
    X_i \mapsto \tilde{\si}(\al_i)
  \]
  Then by definition $I$ is in the kernel of this evaluation map
  so it factors through the quotient by $I$ to give
  an automorphism of $\bQ(\al_0 , \al_1 , \al_2)$ preserving $\bQ$.
\end{proof}
Let us now compute $G_f$. We have the following
\begin{align*}
  \bQ_f = \bQ[\al_0 , \om] 
  = \bQ[\al_0][\om]
  \simeq \frac{\bQ[\al_0][Y]}{(Y^2 + Y + 1)}
  \simeq \frac{\bQ[X][Y]/(X^3 - 2)}{(X^3 - 2 , Y^2 + Y + 1)/(X^3 - 2)}
  \simeq \frac{\bQ[X , Y]}{(X^3 - 2 , Y^2 + Y + 1)}
\end{align*}
where the last isomorphism is the 3rd isomorphism theorem of rings.
Consider the 3-cycle $\si := (\al_0 \,\, \al_1 \,\, \al_2)$.
Knowing $\om = \al_1 / \al_0$ we send $X \mapsto \al_1 , Y \mapsto \omega$.
\begin{cd}
  {\mathbb{Q}[X , Y]} & {\mathbb{Q}[\alpha_0 , \omega]} \\
	{\mathbb{Q}[\alpha_0 , \omega]}
	\arrow["\begin{array}{c} \begin{matrix}X \mapsto \alpha_1 \\ Y \mapsto \omega\end{matrix} \end{array}", from=1-1, to=1-2]
	\arrow["\begin{array}{c} \begin{matrix}X \mapsto \alpha_0 \\ Y \mapsto \omega\end{matrix} \end{array}"', from=1-1, to=2-1]
	\arrow["\simeq"', dashed, from=2-1, to=1-2]
\end{cd}
We get the factoring because $\al_1^3 - 2 = 0 = \om^2 + \om + 1$
and so $\si \in G_f$.
Now consider $\tau := (\al_0 \,\, \al_1)$.
Again, since $\om = \al_1 / \al_0$ we know 
$\tau$ should send $\om \mapsto 1 / \om = \om^2$ so we send 
$X \mapsto \al_0 , Y \mapsto \omega^2$.
\begin{cd}
  {\mathbb{Q}[X , Y]} & {\mathbb{Q}[\alpha_0 , \omega]} \\
	{\mathbb{Q}[\alpha_0 , \omega]}
	\arrow["\begin{array}{c} \begin{matrix}X \mapsto \alpha_1 \\ Y \mapsto \omega^2\end{matrix} \end{array}", from=1-1, to=1-2]
	\arrow["\begin{array}{c} \begin{matrix}X \mapsto \alpha_0 \\ Y \mapsto \omega\end{matrix} \end{array}"', from=1-1, to=2-1]
	\arrow["\simeq"', dashed, from=2-1, to=1-2]
\end{cd}
Again $\al_1^3 - 2 = 0 = (\om^2)^2 + \om^2 + 1$ gives the above factoring
and hence $\tau \in G_f$.
It follows that $G_f$ is the whole of $\AUT\set{\al_0 , \al_1 , \al_2}$.

Symmetry means ``changes that cannot be observed''.
The symmetries of a triangle are the ways you can change the triangle
such that you cannot tell the difference between before and after.
In the same way, $G_f$ are the ways you can swap of roots of $f$
such that as far as $\bQ$ can tell, nothing has changed.
In this example, there is nothing special about $\al_0$;
the whole argument works starting with $\al_1$ or $\al_2$.
The roots are equally ambiguous, which is reflected
in the quantitative fact that $G_f \simeq S_3$.
An example of less ambiguity is $T^3 - 1$.
The roots are $1, \om , \om^2$.
The Galois group of $T^3 - 1$ is cyclic order two generated by 
$\om \mapsto \om^2$.
This reflects the fact that 1 is more special than $\om, \om^2$
whilst the latter cannot be distinguished from each other.
Indeed if one writes $\mu := \om^2$ then $\om = \mu^2$.

Back to $T^3 - 2$. 
Observe that $\bQ \subs \bQ_f^{G_f} := $
the set of elements in $\bQ_f$ fixed by $G_f$.
\textbf{Claim : $\bQ = \bQ_f^{G_f}$.}
Let $x\in \bQ_f$ be fixed by $G_f$.
We approach $\bQ_f$ this time by adding $\om$ first then $\al_0$.
Since $\bQ_f = \bQ[\om][\al_0]$ we can write
\[
  x = \la_0 + \la_1 \al_0 + \la_2 \al_0^2
\]
for $\la_i \in \bQ(\om)$.
Then since $\si(\om) = \om$ we have \[
  x = \si(x) = \la_0 + \la_1 \om \al_0 + \la_2 \om^2 \al_0^2
\]
Since $1 , \al_0 , \al_0^2$ are a $\bQ(\om)$-basis for $\bQ_f$,
we can compare coefficients to get $\la_1 = \la_1 \om$ and $\la_2 = \la_2 \om^2$
This implies $\la_1 = 0 = \la_2$ and so $x \in \bQ(\om)$.
Now $x = \mu_0 + \mu_1 \om$ for $\mu_i \in \bQ$.
Then \[
  x = \tau(x) = \mu_0 + \mu_1 \om^2
  = (\mu_0 - \mu_1) - \mu_1 \om
\]
which implies $\mu_1 = - \mu_1$ and so $\mu_1 = 0$.
We find that $x \in \bQ$.
More generally, given any subgroup $H$ of $G_f$ 
we can compute the \emph{fixed subfield} $\bQ_f^H$.
Here is a diagram of all the subgroups of $G_f$ and their corresponding
fixed subfields.
\begin{cd}[sep = small]
  & {\mathbb{Q}(\alpha_0 ,\alpha_1 ,\alpha_2)} &&&& 1 \\
	{\mathbb{Q}(\omega)} & {\mathbb{Q}(\alpha_0)} & {\mathbb{Q}(\alpha_1)} & {\mathbb{Q}(\alpha_2)} & {\langle(\alpha_0 \,\,\alpha_1 \,\,\alpha_2)\rangle} & {\langle(\alpha_1 \,\,\alpha_2)\rangle} & {\langle(\alpha_0 \,\,\alpha_2)\rangle} & {\langle(\alpha_0 \,\,\alpha_1)\rangle} \\
	& {\mathbb{Q}} &&&& {G_f}
	\arrow[from=1-6, to=2-5]
	\arrow[from=1-6, to=2-6]
	\arrow[from=1-6, to=2-7]
	\arrow[from=1-6, to=2-8]
	\arrow[from=2-1, to=1-2]
	\arrow[from=2-2, to=1-2]
	\arrow[from=2-3, to=1-2]
	\arrow[from=2-4, to=1-2]
	\arrow[from=2-5, to=3-6]
	\arrow[from=2-6, to=3-6]
	\arrow[from=2-7, to=3-6]
	\arrow[from=2-8, to=3-6]
	\arrow[from=3-2, to=2-1]
	\arrow[from=3-2, to=2-2]
	\arrow[from=3-2, to=2-3]
	\arrow[from=3-2, to=2-4]
\end{cd}
The fundamental theorem of Galois theory says
this is all of them.
To be more precise, we make some definitions.

\begin{dfn}[Galois extension]

  Let $K \to L$ be an extension of fields.
  We often identify $K$ with its image in $L$.
  We call it \emph{Galois} when there is a finite group $G \subs \AUT_K L$
  such that $K = L^G$. 
\end{dfn}
The extension earlier $\bQ \subs \bQ(\al_0 , \al_1 , \al_2)$ was an example
of a Galois extension.
\begin{prop}[Fundamental theorem of Galois theory]

  Let $K \to L$ be a Galois extension of fields
  and let $G := \AUT_K L$.
  Consider the following two constructions : 
  \begin{itemize}
    \item Given a subgroup $H \subs G$,
    define $L^H$ as the set of fixed points of $L$ by $H$.
    This defines a field containing the image of $K$.
    \item Given a subfield $M \subs L$ containing $K$,
    define $\AUT_M L$ as the subgroup of $G$ acting trivially on $M$.
  \end{itemize}
  Then we have an order reversing bijection \begin{cd}
    {\set{\text{subextensions }M\subseteq L}} & {\set{\text{subgroups of }\mathrm{Aut}_K L}}
    \arrow["{\mathrm{Aut}_\_ L}", shift left=3, from=1-1, to=1-2]
    \arrow["\simeq"{description}, draw=none, from=1-1, to=1-2]
    \arrow["{L^\_}", shift left=3, from=1-2, to=1-1]
  \end{cd}
\end{prop}
The Galois extension $\bQ_f / \bQ$ is an example of a \emph{solvable} extension.
\begin{dfn}

  Let $K \to L$ be a field extensions.
  We say it is \emph{radical} when
  there exists a chain of extensions 
  \[
    K = L_0 \to L_1 \to \cdots \to L_{n-1} \to L_n = L
  \]
  such that each $L_{i+1} = L_i(\al_i)$
  for some $\al_i$ with $\al_i^{d_i} \in L_i$ for some $d_i > 0$.

  We say a polynomial $f \in K[T]$ is \emph{solvable by radicals}
  when $K_f / K$ is radical.
\end{dfn}

Notice that in the example, that the sequence of groups 
\[
  1 \to \langle(\alpha_0 \,\,\alpha_1 \,\,\alpha_2)\rangle \to G_f
\]
is such that one subgroup is normal in the next
and furthermore that the factor groups are cyclic.
This is an example of a \emph{solvable group}.

\begin{dfn}
  
  Let $G$ be a finite group.
  Then $G$ is called solvable when
  there exists a chain \[
    1 = H_0 \triangleleft H_1 \triangleleft
    \cdots \triangleleft H_{n-1} \triangleleft H_n = G
  \]
  such that $H_{n+1} / H_n$ is cyclic.
\end{dfn}

We will show the following by the end of the course.
\begin{prop}[Characteristization of solvable polynomials]
  
  Let $K$ be a field of characteristic zero and $f \in K[T]$.
  Then $f$ is solvable by radicals iff $G_f$ is solvable.
\end{prop}

\begin{prop}
  The polynomial $T^5 - T - 1 \in \bQ[T]$ has Galois group $S_5$
  and hence is not solvable by radicals.
\end{prop}

% \subsection{Cubic equation}

% Using the fundamental theorem of Galois theory
% we derive the cubic equation.
% Let $f(T) = T^3 + a T^2 + b T + c \in \bQ[T]$.
% Let $S = \set{\al_0 , \al_1 , \al_2} \subs \bC$ be the set of solutions 
% to $f(x) = 0$.
% \begin{itemize}
%   \item Goal : express $\al_i$ using $\bQ , + , - , \times, \div , \sqrt[n]{\_}$
% \end{itemize}
% First, a simplification
% \[
%   f(T - a / 3) = T^3 + 3 p T + 2 q
% \]
% so we will solve $f(T) = T^3 + 3 p T + 2 q$ instead.
% Since we are solving cubics, it is conceivable that
% we may need cube roots of 1 at some point.
% Choose a primitive cube root of unity $\om \neq 1 = \om^3$.
% This is okay for our goal because $\om^2 + \om + 1 = 0$
% so $\om$ can be solved by radicals from $\bQ$.
% We thus change our base field to $K := \bQ(\om)$.
% Consider the elements in $\bC$ : 
% \begin{align*}
%   u &= \al_0 + \om \al_1 + \om^2 \al_2 \\
%   v &= \al_0 + \om^2 \al_1 + \om \al_2
% \end{align*}
% Note that we can express the roots in terms of $u , v$ : 
% using $\om^2 + \om + 1 = 0$ we have
% \begin{align*}
%   u + v = 2 \al_0 + (\om + \om^2) \al_1 + (\om + \om^2) \al_2
%   = 2 \al_0 - \al_1 - \al_2 = 3 \al_0 \\
% \end{align*}
% so $\al_0 = (u + v) / 3$ and similarly $\al_1 = (\om u + \om^2 v) / 3 , 
% \al_2 = (\om^2 u + \om v) / 3$.

% We investigate how elements of $\AUT\set{\al_0 , \al_1 , \al_2}$
% acts on $u$ and $v$.
% Let $\si = (\al_0 \,\,\al_1 \,\,\al_2)$.
% Then \begin{align*}
%   \si(u) = \om^2 u \\
%   \si(v) = \om v
% \end{align*}
% Let $\tau = (\al_1 \,\,\al_2)$ then
% \begin{align*}
%   \tau(u) = v \\
%   \tau(v) = u 
% \end{align*}
% Since $\si , \tau$ generate $\AUT\set{\al_0 , \al_1 , \al_2}$,
% no matter what $G_f$ is it must fix $u^3 + v^3 , u v$
% and hence by the fundamental theorem $u^3 + v^3 , u v \in K$.
% An explicit computation shows : 
% \begin{align*}
%   u v &= - 3^2 p \\
%   u^3 + v^3 &= - 3^3 2 q
% \end{align*}
% Then $u^3$ and $v^3$ are the solutions to $T^2 + 3^3 2 q T - 3^6 p^3 \in K[T]$, 

\section{Finite extensions and the embedding theorem}

\section{Normal extensions}

\section{Separable extensions}

\section{Galois extensions and the fundamental theorem}

\section{Cyclotomic extensions, Kummer extensions, Radical extensions}

\section{Finite fields}

\section{Frobenius lifts and existence of non-solvable quintic}

\printbibliography

\end{document}