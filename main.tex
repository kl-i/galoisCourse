\documentclass{article}
\usepackage[left=1in,right=1in]{geometry}
\usepackage{subfiles}
\usepackage{amsmath, amssymb, stmaryrd, verbatim} % math symbols
\usepackage{amsthm} % thm environment
\usepackage{mdframed} % Customizable Boxes
\usepackage{hyperref,nameref,cleveref,enumitem} % for references, hyperlinks
\usepackage[dvipsnames]{xcolor} % Fancy Colours
\usepackage{mathrsfs} % Fancy font
\usepackage{bbm} % mathbb numerals
\usepackage[bbgreekl]{mathbbol} % for bb Delta
\usepackage{tikz, tikz-cd, float} % Commutative Diagrams
\usetikzlibrary{decorations.pathmorphing} % for squiggly arrows in tikzcd
\usepackage{perpage}
\usepackage{parskip} % So that paragraphs look nice
\usepackage{ifthen,xargs} % For defining better commands
\usepackage[T1]{fontenc}
\usepackage[utf8]{inputenc}
\usepackage{tgpagella}
\usepackage{cancel}

% Bibliography
% \usepackage{url}
\usepackage[backend=biber,
            isbn=false,
            doi=false,
            giveninits=true,
            style=alphabetic,
            useprefix=true,
            maxcitenames=4,
            maxbibnames=4,
            sorting=nyt,
            citestyle=alphabetic]{biblatex}

% Shortcuts

% % Local to this project
\setcounter{secnumdepth}{4}
\renewcommand\labelitemi{--} % Makes itemize use dashes instead of bullets
\newcommand{\bbrkt}[1]{\llbracket #1 \rrbracket}
\newcommand{\IND}{\mathrm{Ind}}
\newcommand{\AFF}{\mathrm{Aff}}
\newcommand{\FSCH}{\mathrm{fSch}}
\newcommand{\SCH}{\mathrm{Sch}}
\DeclareMathOperator{\SPEC}{Spec}
\DeclareMathOperator{\QCOH}{QCoh}
\DeclareMathOperator{\INDCOH}{IndCoh}
\newcommand{\DR}{\mathrm{dR}}
\newcommand{\SYM}{\mathrm{Sym}}
\newcommand{\HOM}{\mathrm{Hom}}
\newcommand{\MAP}{\mathrm{Map}}
\newcommand{\CALG}{\mathrm{cAlg}}
\DeclareMathOperator{\GL}{GL}
\DeclareMathOperator{\LIE}{Lie}
\newcommand{\AD}{\mathrm{Ad}}
\DeclareMathOperator{\MAX}{Max}
\DeclareMathOperator{\FROB}{Frob}
\newcommand{\SEP}{\mathrm{sep}}
\newcommand{\LOC}{\mathrm{Loc}}
\DeclareMathOperator{\PERV}{Perv}
\newcommand{\RED}{\mathrm{red}}
\newcommand{\ZAR}{\mathrm{Zar}}
\newcommand{\FPPF}{\mathrm{fppf}}
\newcommand{\ET}{\text{ét}}
\DeclareMathOperator{\SPF}{Spf}
\newcommand{\CTS}{\mathrm{cts}}
\newcommand{\INF}{\mathrm{Inf}}
\DeclareMathOperator{\STK}{Stk}
\DeclareMathOperator{\PSTK}{PStk}
\DeclareMathOperator{\SSET}{sSet}
\DeclareMathOperator{\BUN}{Bun}
\DeclareMathOperator{\PSH}{PSh}
\DeclareMathOperator{\SH}{Sh}
\newcommand{\FPQC}{\mathrm{fpqc}}
\DeclareMathOperator{\SPH}{Sph}
\newcommand{\FT}{\mathrm{ft}}
\DeclareMathOperator{\CRYS}{Crys}
\DeclareMathOperator{\GR}{Gr}
\DeclareMathOperator{\REP}{Rep}
\newcommand{\HITCH}{\mathrm{Hitch}}
\newcommand{\CRIT}{\mathrm{crit}}
\DeclareMathOperator{\LOCSYS}{LocSys}
\newcommand{\GRPD}{\mathrm{Grpd}}
\DeclareMathOperator{\FUN}{Fun}
\newcommand{\HECKE}{\mathrm{Hecke}}
\newcommand{\TRIV}{\mathrm{Triv}}
\newcommand{\REGLUE}{\mathrm{ReGlue}}
\newcommand{\LVL}{\mathrm{lvl}}
\newcommand{\PT}{\mathrm{pt}}
\DeclareMathOperator{\COH}{Coh}
\DeclareMathSymbol\bbDe  \mathord{bbold}{"01} % blackboard Delta
\newcommand{\CL}{\mathrm{cl}}
\newcommand{\OPER}{\operatorname{Oper}}
\DeclareMathOperator{\PERF}{Perf}
\DeclareMathOperator{\INV}{Inv}
\newcommand{\PIC}{\mathrm{Pic}}
\newcommand{\RAT}{\mathrm{Rat}}
\newcommand{\DIV}{\mathrm{Div}}
\newcommand{\SPA}{\mathrm{Spa}\,}
\newcommand{\SPD}{\mathrm{Spd}\,}
\newcommand{\LA}{\mathrm{la}}
\newcommand{\SNORM}{\mathrm{SNorm}}
\newcommand{\NORM}{\mathrm{Norm}}
\newcommand{\BAN}{\mathrm{Ban}}
\newcommand{\CYC}{\mathrm{cyc}}


% % Misc
\newcommand{\brkt}[1]{\left(#1\right)}
\newcommand{\sqbrkt}[1]{\left[#1\right]}
\newcommand{\dash}{\text{-}}

% % Logic
\renewcommand{\implies}{\Rightarrow}
\renewcommand{\iff}{\Leftrightarrow}
\newcommand{\limplies}{\Leftarrow}
\newcommand{\NOT}{\neg\,}
\newcommand{\AND}{\, \land \,}
\newcommand{\OR}{\, \lor \,}
\newenvironment{forward}{($\implies$)}{}
\newenvironment{backward}{($\limplies$)}{}

% % Sets
\DeclareMathOperator{\supp}{supp}
\newcommand{\set}[1]{\left\{#1\right\}}
\newcommand{\st}{\text{ s.t. }}
\newcommand{\minus}{\setminus}
\newcommand{\subs}{\subseteq}
\newcommand{\ssubs}{\subsetneq}
\newcommand{\sups}{\supseteq}
\newcommand{\ssups}{\supset}
\DeclareMathOperator{\im}{Im}
\newcommand{\nothing}{\varnothing}
\DeclareMathOperator{\join}{\sqcup}
\DeclareMathOperator{\meet}{\sqcap}

% % Greek 
\newcommand{\al}{\alpha}
\newcommand{\be}{\beta}
\newcommand{\ga}{\gamma}
\newcommand{\de}{\delta}
\newcommand{\ep}{\varepsilon}
\newcommand{\ph}{\varphi}
\newcommand{\io}{\iota}
\newcommand{\ka}{\kappa}
\newcommand{\la}{\lambda}
\newcommand{\om}{\omega}
\newcommand{\si}{\sigma}

\newcommand{\Ga}{\Gamma}
\newcommand{\De}{\Delta}
\newcommand{\Th}{\Theta}
\newcommand{\La}{\Lambda}
\newcommand{\Si}{\Sigma}
\newcommand{\Om}{\Omega}

% % Mathbb
\newcommand{\bA}{\mathbb{A}}
\newcommand{\bB}{\mathbb{B}}
\newcommand{\bC}{\mathbb{C}}
\newcommand{\bD}{\mathbb{D}}
\newcommand{\bE}{\mathbb{E}}
\newcommand{\bF}{\mathbb{F}}
\newcommand{\bG}{\mathbb{G}}
\newcommand{\bH}{\mathbb{H}}
\newcommand{\bI}{\mathbb{I}}
\newcommand{\bJ}{\mathbb{J}}
\newcommand{\bK}{\mathbb{K}}
\newcommand{\bL}{\mathbb{L}}
\newcommand{\bM}{\mathbb{M}}
\newcommand{\bN}{\mathbb{N}}
\newcommand{\bO}{\mathbb{O}}
\newcommand{\bP}{\mathbb{P}}
\newcommand{\bQ}{\mathbb{Q}}
\newcommand{\bR}{\mathbb{R}}
\newcommand{\bS}{\mathbb{S}}
\newcommand{\bT}{\mathbb{T}}
\newcommand{\bU}{\mathbb{U}}
\newcommand{\bV}{\mathbb{V}}
\newcommand{\bW}{\mathbb{W}}
\newcommand{\bX}{\mathbb{X}}
\newcommand{\bY}{\mathbb{Y}}
\newcommand{\bZ}{\mathbb{Z}}
\newcommand{\id}{\mathbbm{1}}

% % Mathcal
\newcommand{\cA}{\mathcal{A}}
\newcommand{\cB}{\mathcal{B}}
\newcommand{\cC}{\mathcal{C}}
\newcommand{\cD}{\mathcal{D}}
\newcommand{\cE}{\mathcal{E}}
\newcommand{\cF}{\mathcal{F}}
\newcommand{\cG}{\mathcal{G}}
\newcommand{\cH}{\mathcal{H}}
\newcommand{\cI}{\mathcal{I}}
\newcommand{\cJ}{\mathcal{J}}
\newcommand{\cK}{\mathcal{K}}
\newcommand{\cL}{\mathcal{L}}
\newcommand{\cM}{\mathcal{M}}
\newcommand{\cN}{\mathcal{N}}
\newcommand{\cO}{\mathcal{O}}
\newcommand{\cP}{\mathcal{P}}
\newcommand{\cQ}{\mathcal{Q}}
\newcommand{\cR}{\mathcal{R}}
\newcommand{\cS}{\mathcal{S}}
\newcommand{\cT}{\mathcal{T}}
\newcommand{\cU}{\mathcal{U}}
\newcommand{\cV}{\mathcal{V}}
\newcommand{\cW}{\mathcal{W}}
\newcommand{\cX}{\mathcal{X}}
\newcommand{\cY}{\mathcal{Y}}
\newcommand{\cZ}{\mathcal{Z}}

% % Mathfrak
\newcommand{\f}[1]{\mathfrak{#1}}

% % Mathrsfs
\newcommand{\s}[1]{\mathscr{#1}}

% % Category Theory
\DeclareMathOperator{\obj}{Obj}
\DeclareMathOperator{\END}{End}
\DeclareMathOperator{\AUT}{Aut}
\newcommand{\CAT}{\mathrm{Cat}}
\newcommand{\SET}{\mathrm{Set}}
\newcommand{\TOP}{\mathrm{Top}}
\newcommand{\MON}{\mathrm{Mon}}
\newcommand{\GRP}{\mathrm{Grp}}
\newcommand{\AB}{\mathrm{Ab}}
\newcommand{\RING}{\mathrm{Ring}}
\newcommand{\CRING}{\mathrm{CRing}}
\newcommand{\MOD}{\mathrm{Mod}}
\newcommand{\VEC}{\mathrm{Vec}}
\newcommand{\ALG}{\mathrm{Alg}}
\newcommand{\ORD}{\mathrm{Ord}}
\newcommand{\POSET}{\mathbf{PoSet}}
\newcommand{\map}[2]{\yrightarrow[#2][2.5pt]{#1}[-1pt]}
\newcommand{\iso}[1][]{\cong_{#1}}
\newcommand{\OP}{\mathrm{op}}
\newcommand{\darrow}{\downarrow}
\newcommand{\LIM}{\varprojlim}
\newcommand{\COLIM}{\varinjlim}
\DeclareMathOperator{\coker}{coker}
\newcommand{\fall}[2]{\downarrow_{#2}^{#1}}
\newcommand{\lift}[2]{\uparrow_{#1}^{#2}}

% % Algebra
\newcommand{\nsub}{\trianglelefteq}
\newcommand{\inv}{{-1}}
\newcommand{\dvd}{\,|\,}
\DeclareMathOperator{\ev}{ev}

% % Analysis
\newcommand{\abs}[1]{\left\vert #1 \right\vert}
\newcommand{\norm}[1]{\left\Vert #1 \right\Vert}
\renewcommand{\bar}[1]{\overline{#1}}
\newcommand{\<}{\langle}
\renewcommand{\>}{\rangle}
\renewcommand{\hat}[1]{\widehat{#1}}
\renewcommand{\check}[1]{\widecheck{#1}}
\newcommand{\dsum}[2]{\sum_{#1}^{#2}}
\newcommand{\dprod}[2]{\prod_{#1}^{#2}}
\newcommand{\del}[2]{\frac{\partial#1}{\partial#2}}
\newcommand{\res}[2]{{% we make the whole thing an ordinary symbol
  \left.\kern-\nulldelimiterspace % automatically resize the bar with \right
  #1 % the function
  %\vphantom{\big|} % pretend it's a little taller at normal size
  \right|_{#2} % this is the delimiter
  }}

% % Galois
\DeclareMathOperator{\GAL}{Gal}
\DeclareMathOperator{\ORB}{Orb}
\DeclareMathOperator{\STAB}{Stab}
\newcommand{\emb}[3]{\mathrm{Emb}_{#1}(#2, #3)}
\newcommand{\Char}[1]{\mathrm{Char}#1}

%% code from mathabx.sty and mathabx.dcl to get some symbols from mathabx
\DeclareFontFamily{U}{mathx}{\hyphenchar\font45}
\DeclareFontShape{U}{mathx}{m}{n}{
      <5> <6> <7> <8> <9> <10>
      <10.95> <12> <14.4> <17.28> <20.74> <24.88>
      mathx10
      }{}
\DeclareSymbolFont{mathx}{U}{mathx}{m}{n}
\DeclareFontSubstitution{U}{mathx}{m}{n}
\DeclareMathAccent{\widecheck}{0}{mathx}{"71}

% Arrows with text above and below with adjustable displacement
% (Stolen from Stackexchange)
\newcommandx{\yaHelper}[2][1=\empty]{
\ifthenelse{\equal{#1}{\empty}}
  % no offset
  { \ensuremath{ \scriptstyle{ #2 } } } 
  % with offset
  { \raisebox{ #1 }[0pt][0pt]{ \ensuremath{ \scriptstyle{ #2 } } } }  
}

\newcommandx{\yrightarrow}[4][1=\empty, 2=\empty, 4=\empty, usedefault=@]{
  \ifthenelse{\equal{#2}{\empty}}
  % there's no text below
  { \xrightarrow{ \protect{ \yaHelper[ #4 ]{ #3 } } } } 
  % there's text below
  {
    \xrightarrow[ \protect{ \yaHelper[ #2 ]{ #1 } } ]
    { \protect{ \yaHelper[ #4 ]{ #3 } } } 
  } 
}

% xcolor
\definecolor{darkgrey}{gray}{0.10}
\definecolor{lightgrey}{gray}{0.30}
\definecolor{slightgrey}{gray}{0.80}
\definecolor{softblue}{RGB}{30,100,200}

% hyperref
\hypersetup{
      colorlinks = true,
      linkcolor = {softblue},
      citecolor = {blue}
}

\newcommand{\link}[1]{\hypertarget{#1}{}}
\newcommand{\linkto}[2]{\hyperlink{#1}{#2}}

% Perpage
\MakePerPage{footnote}

% Theorems

% % custom theoremstyles
\newtheoremstyle{definitionstyle}
{5pt}% above thm
{0pt}% below thm
{}% body font
{}% space to indent
{\bf}% head font
{\vspace{1mm}}% punctuation between head and body
{\newline}% space after head
{\thmname{#1}\thmnote{\,\,--\,\,#3}}

\newtheoremstyle{exercisestyle}%
{5pt}% above thm
{0pt}% below thm
{\it}% body font
{}% space to indent
{\it}% head font
{.}% punctuation between head and body
{ }% space after head
{\thmname{#1}\thmnote{ (#3)}}

\newtheoremstyle{examplestyle}%
{5pt}% above thm
{0pt}% below thm
{\it}% body font
{}% space to indent
{\it}% head font
{.}% punctuation between head and body
{\newline}% space after head
{\thmname{#1}\thmnote{ (#3)}}

\newtheoremstyle{remarkstyle}%
{5pt}% above thm
{0pt}% below thm
{}% body font
{}% space to indent
{\it}% head font
{.}% punctuation between head and body
{ }% space after head
{\thmname{#1}\thmnote{\,\,--\,\,#3}}

\newtheoremstyle{questionstyle}%
{5pt}% above thm
{0pt}% below thm
{}% body font
{}% space to indent
{\it}% head font
{?}% punctuation between head and body
{ }% space after head
{\thmname{#1}\thmnote{\,\,--\,\,#3}}

% Custom Environments

% % Theorem environments

\theoremstyle{definitionstyle}
\newmdtheoremenv[
    linewidth = 2pt,
    leftmargin = 0pt,
    rightmargin = 0pt,
    linecolor = darkgrey,
    topline = false,
    bottomline = false,
    rightline = false,
    footnoteinside = true
]{dfn}{Definition}
\newmdtheoremenv[
    linewidth = 2 pt,
    leftmargin = 0pt,
    rightmargin = 0pt,
    linecolor = darkgrey,
    topline = false,
    bottomline = false,
    rightline = false,
    footnoteinside = true
]{prop}{Proposition}
\newmdtheoremenv[
    linewidth = 2 pt,
    leftmargin = 0pt,
    rightmargin = 0pt,
    linecolor = darkgrey,
    topline = false,
    bottomline = false,
    rightline = false,
    footnoteinside = true
]{cor}{Corollary}

\theoremstyle{exercisestyle}
\newmdtheoremenv[
    linewidth = 0.7 pt,
    leftmargin = 20pt,
    rightmargin = 0pt,
    linecolor = darkgrey,
    topline = false,
    bottomline = false,
    rightline = false,
    footnoteinside = true
]{ex}{Exercise}
\newmdtheoremenv[
    linewidth = 0.7 pt,
    leftmargin = 20pt,
    rightmargin = 0pt,
    linecolor = darkgrey,
    topline = false,
    bottomline = false,
    rightline = false,
    footnoteinside = true
]{lem}{Lemma}

\theoremstyle{examplestyle}
\newmdtheoremenv[
    linewidth = 0.7 pt,
    leftmargin = 0pt,
    rightmargin = 0pt,
    linecolor = darkgrey,
    topline = false,
    bottomline = false,
    rightline = false,
    footnoteinside = true
]{eg}{Example}
\newmdtheoremenv[
    linewidth = 0.7 pt,
    leftmargin = 0pt,
    rightmargin = 0pt,
    linecolor = darkgrey,
    topline = false,
    bottomline = false,
    rightline = false,
    footnoteinside = true
]{ceg}{Counter Example}

\theoremstyle{remarkstyle}
\newtheorem{rmk}{Remark}

\theoremstyle{questionstyle}
\newtheorem{question}{Question}

\newenvironment{proof1}{
  \begin{proof}\renewcommand\qedsymbol{$\blacksquare$}
}{
  \end{proof}
} % Proofs ending with black qedsymbol 

% % tikzcd diagram 
\newenvironment{cd}{
    \begin{figure}[H]
    \centering
    \begin{tikzcd}
}{
    \end{tikzcd}
    \end{figure}
}

% tikzcd
% % Substituting symbols for arrows in tikz comm-diagrams.
\tikzset{
  symbol/.style={
    draw=none,
    every to/.append style={
      edge node={node [sloped, allow upside down, auto=false]{$#1$}}}
  }
}

\addbibresource{mybib.bib}

\begin{document}

\title{A twelve hours course in Galois theory}

\author{Ken Lee}
\date{Autumn 2024}
\maketitle

\tableofcontents

\section{The Galois correspondence}

We show an example of the Galois correspondence.
Consider the polynomial $f(T) = T^3 - 2 \in \bQ[T]$.
Let $\al_0 , \al_1 , \al_2 \in \bC$ be the roots of $f$.
\[
  \text{Slogan : Galois theory studies the ``symmetries'' of 
  roots of polynomials}
\]
To make this precise, let us first investigate the field obtained
by chucking in $\al_0 , \al_1 , \al_2$ to $\bQ$.
Define \[
  \bQ_f := \bQ(\al_0 , \al_1 , \al_2) := 
  \text{ smallest field in $\bC$ containing $\bQ , \al_0 , \al_1 ,\al_2$}
\]
\textbf{Question 0 : What does $\bQ_f$ look like?}
We try to describe $\bQ(\al_0)$ first.
Consider the map $T \mapsto \al_0$ 
\begin{cd}
	{\mathbb{Q}[T]} & {\mathbb{C}} \\
	{\mathbb{Q}(\alpha_0)}
	\arrow["{T \mapsto \al_0}", from=1-1, to=1-2]
	\arrow[dashed, from=1-1, to=2-1]
	\arrow["\subseteq"', from=2-1, to=1-2]
\end{cd}
The image is $\bQ[\al_0]$ the collection of polynomial expressions
in $\al_0$ with coefficients in $\bQ$.
Since $f \in \bQ[T]$ is irreducible\footnote{
  Can be checked by Eisenstein's criterion.
  Alternatively, a cubic over $\bQ$ is reducible iff
  it has a root in $\bQ$.
  This can be checked to be impossible by brute force.
}
we have $\bQ[\al_0] = \bQ[T] / (f)$
and hence this has a $\bQ$-basis $1 , \al_0 , \al_0^2$.
\begin{itemize}
  \item \textit{
    Exercise 1 :  show that for a field $K$ and 
    an $K$-algebra $A$ which is finite dimensional 
    as a $K$-vector space and an integral domain,
    $A$ must be field.
  }
\end{itemize}
It follows that $\bQ[\al_0]$ is a field and hence
\[
  \bQ[\al_0] = \bQ(\al_0)
\]
Now we do a trick by observing that \[
  \brkt{\frac{\al_1}{\al_0}}^3 = 2 / 2 = 1
\]
Later on, we will give a way of checking when a polynomial has repeated roots
so assume for now that all $\al_0 , \al_1 , \al_2$ are distinct.
Then we get $\al_1 = \al_0 \om$ for some $\om \neq 1 = \om^3$,
and similarly $\al_2 = \al_0 \om^2$.
The $\om, \om^2$ here are called a \emph{primitive cube roots of unity}.
They are both roots of the polynomial $T^2 + T + 1 \in \bQ[T]$.
In the next section, we will be able to show that $\om \notin \bQ(\alpha_0)$.
Taking this for granted for now,
$T^2 + T + 1$ does not have a root in $\bQ(\al_0)$,
so it is irreducible in $\bQ(\al_0)[T]$.
It follows that \[
  \bQ[\al_0 , \om] \simeq \bQ[\al_0][T] / (T^2 + T + 1)
\]
As a $\bQ[\al_0]$-vector space, this has dimension two and
hence is again a field by Exercise 1. 
We deduce 
\textbf{Answer 0 :} \[
  \bQ(\al_0 , \al_1 , \al_2) = \bQ[\al_0 , \om] = \bQ[\al_0 , \al_1 ,\al_2]
\]

We now define \emph{the Galois group of $f$} as \[
  G_f := \AUT_\bQ \bQ(\al_0 , \al_1 , \al_2)
  := \set{\si : \bQ_f \to \bQ_f \st \si \text{ ring morphism and }
  \forall\,\la \in \bQ\,,\,\si(\la) = \la}
\]
\textbf{Question 1 : Why is this the ``symmetries'' of $\al_0 , \al_1 ,\al_2$?}
Observation : any $\si \in G_f$ must permute $\set{\al_0 , \al_1 , \al_2}$.
This is \textbf{\emph{the} trick} that underlies Galois theory : \[
  f(\si(\al_i)) = (\si(\al_i))^3 - 2 = \si(\al_i^3 - 2) = 0
\]
Hence we have a well-define group morphism \[
  G_f \to \AUT\set{\al_0 , \al_1 , \al_2}
\]
Since $\bQ_f = \bQ[\al_0 , \al_1 ,\al_2]$ any $\si \in G_f$
is determined by what it does on $\al_i$ hence the above morphism is injective.
\textbf{Answer 1 : The above morphism defines an isomorphism}
\[
  G_f \simeq \set{
    \si \in \AUT\set{\al_0 ,\al_1 ,\al_2} \st 
    \forall g \in \bQ[X_0 , X_1 , X_2] \,,
    g(\al_0 , \al_1 ,\al_2) = 0 \implies
    g(\si(\al_0) , \si(\al_1) , \si(\al_2)) = 0
  }
\]
\textbf{in other words, $G_f$ is the permutations of roots of $f$
which preserves all algebraic relations over $\bQ$.}
\begin{proof}
  $\bQ[\al_0 , \al_1 , \al_2]$ is precisely the image of 
  the evaluation map \[
    \bQ[X_0 , X_1 , X_2] \to \bQ[\al_0 , \al_1 , \al_2] \,,\,X_i \mapsto \al_i
  \]
  It follows that $
    \bQ[\al_0 , \al_1 , \al_2] \simeq 
    \bQ[X_0 , X_1 , X_2] / I
  $ where $I$ is the set of polynomials $g(X_0 , X_1 , X_2)$ with
  $g(\al_0 , \al_1 , \al_2)$.
  From this, it is clear that $G_f$ lands inside the RHS.
  Now given $\tilde{\si}$ in RHS,
  one can evaluate \[
    \bQ[X_0 , X_1 , X_2] \to \bQ[\al_0 , \al_1 ,\al_2] \,,\, 
    X_i \mapsto \tilde{\si}(\al_i)
  \]
  Then by definition $I$ is in the kernel of this evaluation map
  so it factors through the quotient by $I$ to give
  an automorphism of $\bQ(\al_0 , \al_1 , \al_2)$ preserving $\bQ$.
\end{proof}
Let us now compute $G_f$. We have the following
\begin{align*}
  \bQ_f = \bQ[\al_0 , \om] 
  = \bQ[\al_0][\om]
  \simeq \frac{\bQ[\al_0][Y]}{(Y^2 + Y + 1)}
  \simeq \frac{\bQ[X][Y]/(X^3 - 2)}{(X^3 - 2 , Y^2 + Y + 1)/(X^3 - 2)}
  \simeq \frac{\bQ[X , Y]}{(X^3 - 2 , Y^2 + Y + 1)}
\end{align*}
where the last isomorphism is the 3rd isomorphism theorem of rings.
Consider the 3-cycle $\si := (\al_0 \,\, \al_1 \,\, \al_2)$.
Knowing $\om = \al_1 / \al_0$ we send $X \mapsto \al_1 , Y \mapsto \omega$.
\begin{cd}
  {\mathbb{Q}[X , Y]} & {\mathbb{Q}[\alpha_0 , \omega]} \\
	{\mathbb{Q}[\alpha_0 , \omega]}
	\arrow["\begin{array}{c} \begin{matrix}X \mapsto \alpha_1 \\ Y \mapsto \omega\end{matrix} \end{array}", from=1-1, to=1-2]
	\arrow["\begin{array}{c} \begin{matrix}X \mapsto \alpha_0 \\ Y \mapsto \omega\end{matrix} \end{array}"', from=1-1, to=2-1]
	\arrow["\simeq"', dashed, from=2-1, to=1-2]
\end{cd}
We get the factoring because $\al_1^3 - 2 = 0 = \om^2 + \om + 1$
and so $\si \in G_f$.
Now consider $\tau := (\al_0 \,\, \al_1)$.
Again, since $\om = \al_1 / \al_0$ we know 
$\tau$ should send $\om \mapsto 1 / \om = \om^2$ so we send 
$X \mapsto \al_0 , Y \mapsto \omega^2$.
\begin{cd}
  {\mathbb{Q}[X , Y]} & {\mathbb{Q}[\alpha_0 , \omega]} \\
	{\mathbb{Q}[\alpha_0 , \omega]}
	\arrow["\begin{array}{c} \begin{matrix}X \mapsto \alpha_1 \\ Y \mapsto \omega^2\end{matrix} \end{array}", from=1-1, to=1-2]
	\arrow["\begin{array}{c} \begin{matrix}X \mapsto \alpha_0 \\ Y \mapsto \omega\end{matrix} \end{array}"', from=1-1, to=2-1]
	\arrow["\simeq"', dashed, from=2-1, to=1-2]
\end{cd}
Again $\al_1^3 - 2 = 0 = (\om^2)^2 + \om^2 + 1$ gives the above factoring
and hence $\tau \in G_f$.
It follows that $G_f$ is the whole of $\AUT\set{\al_0 , \al_1 , \al_2}$.

Symmetry means ``changes that cannot be observed''.
The symmetries of a triangle are the ways you can change the triangle
such that you cannot tell the difference between before and after.
In the same way, $G_f$ are the ways you can swap of roots of $f$
such that as far as $\bQ$ can tell, nothing has changed.
In this example, there is nothing special about $\al_0$;
the whole argument works starting with $\al_1$ or $\al_2$.
The roots are equally ambiguous, which is reflected
in the quantitative fact that $G_f \simeq S_3$.
An example of less ambiguity is $T^3 - 1$.
The roots are $1, \om , \om^2$.
The Galois group of $T^3 - 1$ is cyclic order two generated by 
$\om \mapsto \om^2$.
This reflects the fact that 1 is more special than $\om, \om^2$
whilst the latter cannot be distinguished from each other.
Indeed if one writes $\mu := \om^2$ then $\om = \mu^2$.

Back to $T^3 - 2$. 
Observe that $\bQ \subs \bQ_f^{G_f} := $
the set of elements in $\bQ_f$ fixed by $G_f$.
\textbf{Claim : $\bQ = \bQ_f^{G_f}$.}
Let $x\in \bQ_f$ be fixed by $G_f$.
We approach $\bQ_f$ this time by adding $\om$ first then $\al_0$.
Since $\bQ_f = \bQ[\om][\al_0]$ we can write
\[
  x = \la_0 + \la_1 \al_0 + \la_2 \al_0^2
\]
for $\la_i \in \bQ(\om)$.
Then since $\si(\om) = \om$ we have \[
  x = \si(x) = \la_0 + \la_1 \om \al_0 + \la_2 \om^2 \al_0^2
\]
Since $1 , \al_0 , \al_0^2$ are a $\bQ(\om)$-basis for $\bQ_f$,
we can compare coefficients to get $\la_1 = \la_1 \om$ and $\la_2 = \la_2 \om^2$
This implies $\la_1 = 0 = \la_2$ and so $x \in \bQ(\om)$.
Now $x = \mu_0 + \mu_1 \om$ for $\mu_i \in \bQ$.
Then \[
  x = \tau(x) = \mu_0 + \mu_1 \om^2
  = (\mu_0 - \mu_1) - \mu_1 \om
\]
which implies $\mu_1 = - \mu_1$ and so $\mu_1 = 0$.
We find that $x \in \bQ$.
More generally, given any subgroup $H$ of $G_f$ 
we can compute the \emph{fixed subfield} $\bQ_f^H$.
Here is a diagram of all the subgroups of $G_f$ and their corresponding
fixed subfields.
\begin{cd}[sep = small]
  & {\mathbb{Q}(\alpha_0 ,\alpha_1 ,\alpha_2)} &&&& 1 \\
	{\mathbb{Q}(\omega)} & {\mathbb{Q}(\alpha_0)} & {\mathbb{Q}(\alpha_1)} & {\mathbb{Q}(\alpha_2)} & {\langle(\alpha_0 \,\,\alpha_1 \,\,\alpha_2)\rangle} & {\langle(\alpha_1 \,\,\alpha_2)\rangle} & {\langle(\alpha_0 \,\,\alpha_2)\rangle} & {\langle(\alpha_0 \,\,\alpha_1)\rangle} \\
	& {\mathbb{Q}} &&&& {G_f}
	\arrow[from=1-6, to=2-5]
	\arrow[from=1-6, to=2-6]
	\arrow[from=1-6, to=2-7]
	\arrow[from=1-6, to=2-8]
	\arrow[from=2-1, to=1-2]
	\arrow[from=2-2, to=1-2]
	\arrow[from=2-3, to=1-2]
	\arrow[from=2-4, to=1-2]
	\arrow[from=2-5, to=3-6]
	\arrow[from=2-6, to=3-6]
	\arrow[from=2-7, to=3-6]
	\arrow[from=2-8, to=3-6]
	\arrow[from=3-2, to=2-1]
	\arrow[from=3-2, to=2-2]
	\arrow[from=3-2, to=2-3]
	\arrow[from=3-2, to=2-4]
\end{cd}
The fundamental theorem of Galois theory says
this is all of them.
To be more precise, we make some definitions.

\begin{dfn}[Galois extension]

  Let $K \to L$ be an extension of fields.
  We often identify $K$ with its image in $L$.
  We call it \emph{Galois} when there is a finite group $G \subs \AUT_K L$
  such that $K = L^G$. 
\end{dfn}
The extension earlier $\bQ \subs \bQ(\al_0 , \al_1 , \al_2)$ was an example
of a Galois extension.
\begin{prop}[The Galois correspondence]

  Let $K \to L$ be a Galois extension of fields
  and let $G := \AUT_K L$.
  Consider the following two constructions : 
  \begin{itemize}
    \item Given a subgroup $H \subs G$,
    define $L^H$ as the set of fixed points of $L$ by $H$.
    This defines a field containing the image of $K$.
    \item Given a subfield $M \subs L$ containing $K$,
    define $\AUT_M L$ as the subgroup of $G$ acting trivially on $M$.
  \end{itemize}
  Then we have an order reversing bijection \begin{cd}
    {\set{\text{subextensions }M\subseteq L}} & {\set{\text{subgroups of }\mathrm{Aut}_K L}}
    \arrow["{\mathrm{Aut}_\_ L}", shift left=3, from=1-1, to=1-2]
    \arrow["\simeq"{description}, draw=none, from=1-1, to=1-2]
    \arrow["{L^\_}", shift left=3, from=1-2, to=1-1]
  \end{cd}
\end{prop}
The Galois extension $\bQ_f / \bQ$ is an example of a \emph{solvable} extension.
\begin{dfn}

  Let $K \to L$ be an extension.
  We say it is \emph{radical} when
  there exists a chain of subextensions 
  \[
    K = L_0 \to L_1 \to \cdots \to L_{n-1} \to L_n = L
  \]
  such that each $L_{i+1} = L_i(\al_i)$
  for some $\al_i$ with $\al_i^{d_i} \in L_i$ for some $d_i > 0$.

  For $f \in K[T]$ we say $f$ is \emph{solvable by radicals} when
  there exists a radical extension $K \to L$ which splits $f$.
\end{dfn}

Notice that in the example, that the sequence of groups 
\[
  1 \to \langle(\alpha_0 \,\,\alpha_1 \,\,\alpha_2)\rangle \to G_f
\]
is such that one subgroup is normal in the next
and furthermore that the factor groups are cyclic.
This is an example of a \emph{solvable group}.

\begin{dfn}
  
  Let $G$ be a finite group.
  Then $G$ is called solvable when
  there exists a chain \[
    1 = H_0 \triangleleft H_1 \triangleleft
    \cdots \triangleleft H_{n-1} \triangleleft H_n = G
  \]
  such that $H_{n+1} / H_n$ is cyclic.
\end{dfn}

We will show the following by the end of the course.
\begin{prop}[Characteristization of solvable polynomials]
  
  Let $K$ be a field of characteristic zero and $f \in K[T]$.
  Then $f$ is solvable by radicals iff $G_f$ is solvable.
\end{prop}

\begin{prop}
  The polynomial $T^5 - T - 1 \in \bQ[T]$ has Galois group $S_5$
  and hence is not solvable by radicals.
\end{prop}

% \subsection{Cubic equation}

% Using the fundamental theorem of Galois theory
% we derive the cubic equation.
% Let $f(T) = T^3 + a T^2 + b T + c \in \bQ[T]$.
% Let $S = \set{\al_0 , \al_1 , \al_2} \subs \bC$ be the set of solutions 
% to $f(x) = 0$.
% \begin{itemize}
%   \item Goal : express $\al_i$ using $\bQ , + , - , \times, \div , \sqrt[n]{\_}$
% \end{itemize}
% First, a simplification
% \[
%   f(T - a / 3) = T^3 + 3 p T + 2 q
% \]
% so we will solve $f(T) = T^3 + 3 p T + 2 q$ instead.
% Since we are solving cubics, it is conceivable that
% we may need cube roots of 1 at some point.
% Choose a primitive cube root of unity $\om \neq 1 = \om^3$.
% This is okay for our goal because $\om^2 + \om + 1 = 0$
% so $\om$ can be solved by radicals from $\bQ$.
% We thus change our base field to $K := \bQ(\om)$.
% Consider the elements in $\bC$ : 
% \begin{align*}
%   u &= \al_0 + \om \al_1 + \om^2 \al_2 \\
%   v &= \al_0 + \om^2 \al_1 + \om \al_2
% \end{align*}
% Note that we can express the roots in terms of $u , v$ : 
% using $\om^2 + \om + 1 = 0$ we have
% \begin{align*}
%   u + v = 2 \al_0 + (\om + \om^2) \al_1 + (\om + \om^2) \al_2
%   = 2 \al_0 - \al_1 - \al_2 = 3 \al_0 \\
% \end{align*}
% so $\al_0 = (u + v) / 3$ and similarly $\al_1 = (\om u + \om^2 v) / 3 , 
% \al_2 = (\om^2 u + \om v) / 3$.

% We investigate how elements of $\AUT\set{\al_0 , \al_1 , \al_2}$
% acts on $u$ and $v$.
% Let $\si = (\al_0 \,\,\al_1 \,\,\al_2)$.
% Then \begin{align*}
%   \si(u) = \om^2 u \\
%   \si(v) = \om v
% \end{align*}
% Let $\tau = (\al_1 \,\,\al_2)$ then
% \begin{align*}
%   \tau(u) = v \\
%   \tau(v) = u 
% \end{align*}
% Since $\si , \tau$ generate $\AUT\set{\al_0 , \al_1 , \al_2}$,
% no matter what $G_f$ is it must fix $u^3 + v^3 , u v$
% and hence by the fundamental theorem $u^3 + v^3 , u v \in K$.
% An explicit computation shows : 
% \begin{align*}
%   u v &= - 3^2 p \\
%   u^3 + v^3 &= - 3^3 2 q
% \end{align*}
% Then $u^3$ and $v^3$ are the solutions to $T^2 + 3^3 2 q T - 3^6 p^3 \in K[T]$, 

\section{Finite extensions and the embedding theorem}

We saw in the previous section that 
$\om \in \bQ(\al_0)$ precisely when there is a solution to $T^2 + T + 1$
inside $\bQ(\al_0)$.
Accordingly, there is no copy of $\bQ(\om)$ inside $\bQ(\al_0)$.
This section investigates this phenomenon.
We didn't formally define field extensions last time.

\begin{dfn}
  A field extension is a ring morphism $\io : K \to L$ between fields.
  
  Since fields have no non-trivial ideals,
  any field extension $\io : K \to L$ must be injective.
  When it is clear, we often identify $K$ with its image $\io K$.
  Sometimes we write $L / K$ to say $L$ is an extension of $K$.
\end{dfn}

\begin{eg}
  Here is an example of a field extension from a field to itself.
  Let $\bQ(T) := \mathrm{Frac}\,\bQ[T]$.
  Define $\bQ[T] \to \bQ[T] , T \mapsto T^2$.
  Then this induces a field extension $\bQ(T) \to \bQ(T)$
  where the image of the first copy is $\bQ(T^2)$.
\end{eg}

A basic invariant of a field extension is its degree.

\begin{dfn}[Degree of an extension]
  Let $K \to L$ be a field extension.
  Define its \emph{degree} as $[L : K] := \dim_K L$.
  It is called finite when $[L : K] < \infty$.
\end{dfn}

When proving things about a finite extension $K \to L$,
we will often do so by inducting on $[L : K]$.
The following is useful.

\begin{prop}[Tower law]
  
  Let $K \to L \to N$ be extensions of fields.
  Then $[N : K] = [N : L][L : K]$.
  In particular, a sequence of finite extensions is finite.
\end{prop}
The following argument works for infinite extensions,
though we will mostly be interested in finite extensions.
\begin{proof}
  Let $B_L \subseteq L$ be a $\io_L$-basis and 
  $B_N \subseteq N$ a $\io_N$-basis. 
  The claim is that $B_L B_N := \{ a b \,|\, a \in B_L, b \in B_N\}$
  is a $(\io_N \circ \io_L)$-basis of $N$ and has cardinality $B_L \times B_N$.

  (Cardinality) Let $(a_1, b_1), (a_2, b_2) \in B_L \times B_N$
  such that $ a_1 b_1 = a_2 b_2$. 
  This is then a non-trivial $L$-linear combination of elements in $B_N$,
  contradicting linear independence of $B_N$.
  The cardinality is thus as desired. 

  (Linear Independence) 
  Let $\sum_{(a, b) \in B_L \times B_N} \la_{a,b} a b = 0$
  where $\la_{a,b} \in K$ and only finitely many are non-zero. 
  Then we have 
  $\sum_{b \in B_N} \left(\sum_{a \in B_L} \la_{a,b} a\right) b = 0$,
  giving $\sum_{a \in B_L} \la_{a,b} a = 0$ by linear independence of $B_N$,
  which in turn gives $\la_{a,b} = 0$ by linear independence of $B_L$.
  
  (Spanning) 
  Let $x \in N$. 
  Since $B_N$ is spanning, we have $\sum_{b \in B_N} \la_b b = x$ 
  for some $\la_b \in L$, finitely many non-zero. 
  Then since $B_L$ is spanning, 
  we have $\sum_{a \in B_L} \mu_{a,b} a = \la_b$ for each $b \in N_B$, 
  where $\mu_{a,b} \in K$, finitely many non-zero. 
  So $\sum_{(a,b) \in B_L \times B_N} \mu_{a, b} a b = x$ as desired. 
\end{proof}

\begin{eg}
  Now we can show $\om \notin \bQ(\al_0)$ from the previous section.
  We have $3 = [\bQ(\al_0) : \bQ] = [\bQ(\al_0) : \bQ(\om)] [\bQ(\om) : \bQ]
  = [\bQ(\al_0) : \bQ(\om)] 2$ which is a contradiction because 2 does not
  divide 3.
\end{eg}

\begin{dfn}
  
  Let $K \to L$ be a field extension.
  For $A \subs L$, define $K(A) \subs L$ as the smallest
  subfield of $L$ containing the image of $K$ and $A$.
  We say $K \to L$ is finite type when 
  there exists finite $A \subs L$ with $L = K(A)$.
  In the case of $A = \set{a}$, we write $K(a)$.
  We call extensions of the form $K \to K(a)$ simple.

  Given $a \in L$, one can consider the evaluation 
  ring morphism \[
    \ev_a : K[T] \to L , f(T) \mapsto f(a)
  \]
  We say $a$ is \emph{algebraic over $K$} when
  there exists a non-zero $f$ with $f(a) = 0$,
  i.e. $0 \neq \ker \ev_a$.
  
  We say $K \to L$ is algebraic when
  all $a \in L$ is algebraic over $K$.
\end{dfn}

\begin{prop}[\link{char_ext_simp}{Characteristization of finite simple extensions}]
  Let $K \to L$ be an extension and $a \in L$.
  Then the following are equivalent : 
  \begin{enumerate}
    \item $a$ is algebraic over $K$
    \item $[K(a) : K]$ is finite
    \item $K \to K(a)$ is algebraic.
  \end{enumerate}
\end{prop}
\begin{proof}
  ($1 \implies 2$) 
  We saw in section 1 how to compute $K(a)$.
  Specifically, consider the evaluation map $K[T] \to L , f \mapsto f(a)$
  and let $K[a]$ be its image.
  By assumption, there exists non-zero $f \in K[T]$ with $f(a) = 0$.
  WLOG $\deg f = N \geq 0$.
  Then $1 , a , \dots, a^{N-1}$ is a $K$-spanning set for $K[a]$.
  This implies $K[a]$ is a finite dimensional $K$-vector space
  and hence a field and hence $K[a] = K(a)$.

  ($2 \implies 3$)
  Let $b \in K(a)$.
  Since $[K(a) : K]$ is finite,
  there exists a non-trivial linear combination
  $0 = \sum_{n \geq 0} \la_n b^n$ with $\la_n \in K$,
  which implies $b$ is algebraic over $K$.

  ($3 \implies 1$) trivial.
\end{proof}

\begin{prop}[\link{char_ext_fin}{Characteristization of finite extensions}]
  
  Let $K \to L$ be an extension.
  The following are equivalent : 
  \begin{enumerate}
    \item $[L : K]$ is finite
    \item $K \to L$ is finite type and algebraic 
    \item There exists finite $A \subs L$ such that $L = K(A)$ and
    all $a \in A$ are algebraic.
  \end{enumerate}
\end{prop}
\begin{proof}
  ($1 \implies 2$) Take a $K$-basis and use \linkto{char_ext_simp}{
    the characterization of finite simple extensions
  }.
  ($2 \implies 3$) Clear.
  ($3 \implies 1$)
  Induct on the size of $A$ and use 
  \linkto{char_ext_simp}{
    the characterization of finite simple extensions
  }.

\end{proof}

We are now ready for the main result of this section.

\begin{prop}[Embedding theorem for finite simple extensions]
  
  Let $K \to L$ be an extension and $a \in L$ algebraic over $K$.
  The ideal $\ker \ev_a \subs K[T]$ is generated by a unique
  monic polynomial.
  We call it the \emph{minimal polynomial of $a$ over $K$},
  denoted $\min(a , K)$.
  Let $K \to N$ be another extension.
  Then we have a bijection \[
    \EMB_K(K(a) , N) \simeq \set{b \in N \st \min(a , K) = \min(b , K)},
    \varphi \mapsto \varphi(a)
  \]
  In particular, $\abs{\EMB_K(K(a) , N)} \leq [K(a) : K]$.
  Elements $b \in N$ with $\min(b , K) = \min(a , K)$ are called 
  \emph{Galois conjugates of $a$}.

\end{prop}
\begin{proof}
  We saw $K(a) = K[a] \simeq K[T]/(\min(a , K))$.
  Given $\varphi : K(a) \to N$ a $K$-embedding,
  the composition $K[T] \to K(a) \to N$ is $\ev_{\varphi(a)}$.
  Since $K(a) \to N$ is injective, we have $\ker \ev_{\varphi(a)} = \ker \ev_a$.
  It follows that $\min(\varphi(a) , K) = \min(a , K)$.
  Conversely, given $b \in N$ a Galois conjugate of $a$ we can define the
  $K$-embedding $K(a) \simeq K[T] / (\min(a , K)) = K[T] / (\min(b , K))
  \simeq K(b) \subs N$.
\end{proof}

We will now generalise the above to general finite extensions.
For this, we need to know how embeddings from subextensions
interact with the whole extension.

\begin{prop}[Subextensions partition embeddings]
  
  Let $K \to L \to M$ and $K \to N$ be extensions.
  Then we have a bijection \[
    \bigsqcup_{\io \in \EMB_K(L , N)} \EMB_L(M , N) \map{\sim}{} \EMB_K(M , N)
  \]
  by sending $(L \to N \in \EMB_K(L , N) , M \to N \in \EMB_L(M , N))$
  to $M \to N$ viewed as a $K$-embedding.
\end{prop}
\begin{proof}
  The point is that we have a map $\EMB_K(M , N) \to \EMB_K(L , N)$
  and the fibers over each $\io : L \to N$ is precisely
  the set of $L$-embeddings $M \to N$ where $N$ is viewed as
  an $L$-extension by $\io : L \to N$.
\end{proof}

\begin{prop}[Embedding theorem for finite extensions]
  
  Let $K \to L$ be an extension and $A \subs L$ finite set of algebraic
  generators for $L$ over $K$.
  Let $K \to N$ be another extension and assume that
  for all $a \in A$ the minimal polynomial $\min(a , K)$ splits
  into linear factors in $N[T]$.
  Then \[
    0 < \abs{\EMB_K(L , N)} \leq [L : K]
  \]
  and we have equality if for all $a \in A$ the polynomial $\min(a , K)$
  has no \emph{repeated} roots in $N$. 
\end{prop}
\begin{proof}
  Induct on the cardinality of $A$.
  $A = \nothing$ is trivial so let $a_0 \in A$ and 
  $M := K(A \setminus \set{a_0})$ and assume inductively
  $0 < \EMB_K(M , N) \leq [M : K]$ with equality if
  all for all $a_1 \in A\setminus\set{a_0}$ 
  we have $\min(a_1 , K)$ with no repeated roots
  in $N$.
  Then $L = M(a_0)$.
  We have $\min(a_0, M)$ divides $\min(a_0, K)$ in $M[T]$,
  so $\min(a_0 , M)$ also splits into linear factors in $N[T]$.
  It follows from 
  \linkto{char_ext_simp}{the characterization of finite simple extensions}
  and the tower law
  that
  \[
    0 < \abs{\EMB_K(L , N)}
    = \sum_{\EMB_K(M , N)} \abs{\EMB_M(L , N)}
    \leq \sum_{\EMB_K(M , N)} [L : M]
    \leq [L : M] [M : K] = [L : K]
  \]
  Now assume all $\min(a , K)$ for $a \in A$ split into linear factors in $N$.
  This implies $\min(a_0 , M)$ splits into linear factors in $N$
  so $\abs{\EMB_M(L , N)} = [L : M]$.
  Then the first $\leq$ is an equality and the second is also
  by the induction hypothesis on $M$.

\end{proof}

\section{Normal and separable extensions}

Given an extension $K \to L = K(a_1 , \dots , a_n)$
with $a_i$ algebraic over $K$,
the embedding theorem for finite extensions tells us
how to construct automorphisms of $L$ over $K$.
For the main theorem of Galois theory to hold true,
we need to have the maximum number of automorphisms,
i.e. $\abs{\AUT_K L} = [L : K]$.
The embedding theorem indicates two ways in which this can fail : 
\begin{enumerate}
  \item the polynomials $\min(a_i , K)$ do not split into
  linear factors in $L[X]$
  \item there exists some $a_i$ such that
  $\min(a_i , K)$ has a repeated root in $L$.
\end{enumerate}
These two phenomena are respectively called normality and separability.
Let us illustrate the failure of normality
by focusing on the extension
$\bQ \to \bQ(\al_0)$ from the first section.
Using the embedding theorem for finite simple extensions,
we see that $\si \in \EMB_\bQ(\bQ(\al_0) , \bQ(\al_0))$
correspond to solutions of $T^3 - 2$ in $\bQ(\al_0)$.
There is only $\al_0$:
If there is another root $\tilde{\al_1}$ then
$\tilde{\om} := \tilde{\al_1} / \al_0$ would be a primitive
cube root of unity and $[\bQ(\tilde{\om}) : \bQ] = 2$
which we cannot have as we saw before.
From this, we can see the problem is that 
$\bQ(\al_0) / \bQ$ does not contain
\emph{all} the roots of the polynomial $T^3 - 2$.
More precisely, $T^3 - 2$ does not factorise into linear factors
in $\bQ(\al_0)[T]$.
We can also see this phenomenon in the following way : 
there are three ways of $\bQ$-embedding $\bQ(\al_0)$ inside 
$\bQ(\al_0 , \al_1 , \al_2)$ corresponding to each $\bQ(\al_i)$
and their images are \emph{different}.

\begin{dfn} [\link{dfn:normal}{Normal Extension}] 
  
  Let $K \to L$ be an extension and $f \in K[X]$. 
  Then we say \emph{$L$ splits $f$} when 
  $f$ factorises into linear factors in $L[X]$. 

  Suppose $L/K$ is algebraic. 
  Then it is called \emph{normal} when for all $a \in L$,  
  it contains all the Galois $K$-conjugates of $a$,
  i.e. $L$ splits $\min(a,K)$. 
\end{dfn}

\begin{prop} [\link{thm:split}
  {Splitting Polynomials}]
  
  Let $K$ be a field and $f \in K[X] \setminus K$.
  Then there exists an extension $K \to L$ such that
  $f$ has a root in $L$. 
  In particular, there exists a $K$-extension that splits $f$. 
\end{prop}
\begin{proof}
  Since $f$ is non-constant and $K[X]$ is a UFD, 
  there exists an irreducible $f_1$ that divides $f$.
  Let $L = K[X] / (f_1)$. 
  Then since $f_1$ is irreducible and $K[X]$ is a PID, 
  $L$ is a field and thus a $K$-extension. 
  Note that the image of the monomial $X$ in $L$ is a root of $f_1$,
  and hence a root of $f$.  
  To split $f$, use the above procedure to 
  inductively construct a desired extension.
\end{proof}

\begin{prop} [\link{thm:char_normal}
  {Characterisation of Finite Normal Extensions}]
  
  Let $K \to L$ be a finite extension.
  Then the following are equivalent : 
  \begin{enumerate}
    \item (Contains all Galois $K$-Conjugates) $K \to L$ normal. 
    \item (Contains all Galois $K$-Conjugates of Generators) 
    There exists $A \subseteq L$ a finite set of generators of $K \to L$
    such that for all $a \in A$, 
    $a$ is algebraic over $K$ and $L$ splits $\min(a,K)$. 
    \item (is a Splitting Field) 
    There exists a polynomial $f \in K[X]$ such that
    $L$ splits $f$ and is generated by the roots of $f$ in $L$.
    \item (Image Invariance) For all extensions $K \to N$
    and two $\io_0, \io_1 \in \emb{K}{L}{N}$, 
    $\io_0 L = \io_1 L$.
  \end{enumerate}
\end{prop}
\begin{proof} $(1 \implies 2 \implies 3)$ is clear. 

  $(3 \implies 4)$ The key is that roots of $f$ remain roots of $f$
  under $K$-embeddings. 
  Let $f(X) = \prod_{k = 1}^{\deg f} (X - a_k) \in L[X]$.
  where $a_k \in L$.
  Then $f(X) = \prod_{k = 1}^{\deg f} (X - \io_0(a_k)) \in N[X]$
  For all $a_l$, since $\io_1$ fixes $K$ we get
  \[
    0 = \io_1(f(a_l)) = f(\io_1(a_l))
    = \prod_{k = 1}^{\deg f} (\io_1(a_l) - \io_0(a_k))
  \]
  so there exists $a_k$ such that $\io_1(a_l) = \io_0(a_k)$.
  Since $L = K(a_1,\dots,a_{\deg f})$, 
  this shows that $\io_1 L \subseteq \io_0 L$
  and by symmetry $\io_0 L \subseteq \io_1 L$ as well. 

  $(4 \implies 1)$ Let $a \in L$. 
  Since $(L,\io_L)$ is finite, $\min(a,K)$ exists.
  We do not know if $L$ splits $\min(a,K)$,
  but there exists an extension $L \to M$ such that 
  $M$ splits $\min(a,K)$.
  We seek to show that all Galois $K$-conjugates of $a$ in $M$ 
  are actually in (the image of) $L$ already.
  So let $\al \in M$ be a Galois $K$-conjugate of $a$. 
  We have the following situation. 
  \begin{figure} [H]
    \centering
    \begin{tikzcd}
      K \arrow["\io_L",r] & 
      K(a) \arrow["\subseteq",r] \arrow["\phi_\al",rd] & 
      L \arrow["\io_M",d]\\
      & & M
    \end{tikzcd}
  \end{figure}
  By the \linkto{lem:embed_simp}{embedding theorem for finite simple extensions}, 
  there exists $\phi_\al \in \emb{K}{K(a)}{M}$
  that maps $a \mapsto \al$. 
  Suppose we have an $\io_1 \in \emb{K(a)}{L}{\phi_\al}$.
  Then certainly $\io_1 \in \emb{K}{L}{\io_M \circ \io_L}$. 
  Also, trivially $\io_M \in \emb{K}{L}{\io_M \circ \io_L}$. 
  So $\io_1 L = \io_M L$ implies $\al \in \io_M L$ as desired. 
  It thus suffices to give an $\io_1 \in \emb{K(a)}{L}{\phi_\al}$. 
  Well, since $(L,\io_L)$ is finite, it is also a finite $K(a)$-extension, 
  so it is generated by some finite subset $B$
  whose elements are all algebraic over $K(a)$. 
  Then we can extend $M$ so that it splits all $\min(b,K(a))$ for $b \in B$. 
  Thus by the \linkto{thm:embed}{embedding theorem}, 
  we have an $\io_1 \in \emb{K(a)}{L}{\phi_\al}$. 
\end{proof}

Now let us discuss separability.
As we will see,
existence of inseparable irreducible polynomials
is linked with the \emph{characteristic} of the base field $K$.
This implies that 
in terms of finding an insolvable quintic over $\bQ$,
the problem of inseparable minimal polynomials never happens.

\begin{dfn} [\link{dfn:sep_poly}
  {Separable Polynomial, Separable extension}]

  $f$ is said to be \emph{separable} when 
  for all $K$-extensions in which $f$ splits, $f$ has no repeated roots. 
  If otherwise, $f$ is called \emph{inseparable}. 
  An algebraic extension $K \to L$ is called 
  separable when for all $a \in L$,
  the polynomial $\min(a , K)$ is separable.
\end{dfn}

\begin{prop}[Characterization of separable polymomials using differentials]
  
  Let $K$ be a field and $f = \sum_{0 \leq n} f_n X^n \in K[X]$. 
  The \emph{formal derivative of $f$} is defined to be 
  $f^\prime = \sum_{0 < n} n f_n X^{n-1}$. 
  Then $f$ is separable iff $(f , f^\prime) = 1$.
\end{prop}
Intuition : if $a \in K$, writing $f(X) = \sum_{d \geq 0} \la_d (X - a)^d$,
$a$ is a higher order root iff $\la_0 = \la_1 = 0$.
\begin{proof}
  We will prove $f$ is inseparable iff $(f , f^\prime) \neq 1$.
  Assume $f$ is inseparable.
  Suppose $(f , f^\prime) = 1$.
  Then by the Euclidean algorithm there exists
  $\la, \mu \in K[X]$ such that $\la f + \mu f^\prime = 1$.
  Let $K \to L$ be an extension where $f$ has a repeated root $a$.
  By factoring $f(X) = (X - a)^2 g(X)$ in $L[X]$ and the product rule
  for formal differentiation (which can be proved by induction),
  we see a contradiction \[
    1 = \la(a) f(a) + \mu(a) f^\prime(a) = 0 + 0 = 0
  \]
  Now assume $(f , f^\prime) \neq 1$.
  Let $h \in K[X]$ be the GCD of $f$ and $f^\prime$,
  which is non-constant by assumption.
  Let $K \to L$ be any extension that splits $f$.
  It also splits $h$.
  Let $a \in L$ with $h(a) = 0$.
  We can write $f(X) = (X - a)^d g(X)$ in $L[X]$ for some $d \geq 0$
  and $g(a) \neq 0$.
  Since $h$ divides $f$ we have $f(a) = 0$ so $d \geq 1$.
  Suppose $d = 1$.
  We also have $h$ divides $f^\prime$ yielding a contradiction \[
    0 = f^\prime(a) = g(a) \neq 0
  \]
\end{proof}

To give an example of an inseparable extension,
we need to discuss the notion of the characteristic of a field.

\begin{dfn} [\link{dfn:char_field}
  {Characteristic of a Field}] 
  
  Let $K$ be a field. 
  $\bZ$ is generated by $1$ and ring morphisms must preserve $1$, 
  so there is a unique ring morphism $\bZ \to K$.
  Its image is an ID since $K$ is an ID.
  So by $\bZ$ PID, its kernel is generated by either zero or a (positive) prime. 
  This is defined as the \emph{characteristic of $K$},
  denoted $\Char{K}$.

  More generally, the characteristic of any integral domain $A$
  is defined in the same way.
\end{dfn}

\begin{eg}
  All fields $K$ of characteristic 0 have a unique 
  extension map $\bQ \to K$.
  Similarly, all fields $K$ of characteristic $p > 0$
  have a unique extension map $\bF_p \to K$.
\end{eg}

The following is the root of all interesting phenomena
in positive characteristic.

\begin{prop}[Freshman's dream]
  
  Let $A$ be an $\bF_p$-algebra, i.e. $p = 0$ in $A$,
  and $a , b \in A$.
  Then $(a + b)^p = a^p + b^p$
\end{prop}
\begin{proof}
  The point is that the binomial coefficient 
  $\begin{pmatrix}
    p \\ k
  \end{pmatrix}$
  for $0 < k < p$ is divisible by $p$.
\end{proof}

\begin{eg}
  Consider $K = \bF_p(T) := \mathrm{Frac}\,\bF_p[T]$
  and the polynomial $f(X) = X^p - T \in K[X]$.
  Then by Eisenstein's criterion $f$ is irreducible.
  Let $L := K[X] / (f)$ and $T^{1 / p}$ the image of $X$ in $L$.
  Then in $L[X]$ we have by Freshman's dream \[
    f(X) = X^p - T = X^p - (T^{1 / p})^p
    = (X - T^{1 / p})^p
  \]
  So $f$ is inseparable.
  Notice in that $f^\prime = 0$ so indeed $(f , f^\prime) \neq 1$.
\end{eg}

In fact, we cannot have inseparable extensions in characteristic zero.

\begin{prop}
  Let $K$ be characteristic zero.
  Then any irreducible $f \in K[T]$ is separable.
\end{prop}
\begin{proof}
  $f^\prime$ is either zero or has degree strictly less than $f$.
  WLOG $f$ is monic.
  Then $0 = f^\prime$ implies by looking at the leading coefficient,
  $0 = \deg f$ as elements of $K$,
  contradicting the characteristic of $K$ being zero.
  So $f^\prime \neq 0$.
  But then we must have $(f , f^\prime) = 1$ because
  $\deg f^\prime < \deg f$ implies $f$ cannot divide $f^\prime$.
\end{proof}

% \begin{prop} [\link{thm:char_insep_irr_poly}
%   {Characterisation of Inseparable Irreducible Polynomials}]
  
%   Let $K$ be a field and $f \in K[X]$ irreducible. 
%   Then the following are equivalent : 
%   \begin{enumerate}
%     \item (Repeated Root) $f$ is inseparable. 
%     \item (Intrinsic Definition) $(f,f^\prime) \neq 1$.
%     \item (Another Intrinsic Definition) $f^\prime = 0$.
%     \item (Characteristic Non-zero) 
%     $\Char{K} = p \neq 0$ and 
%     there exists an irreducible separable $g \in K[X]$ with $n > 0$ such that 
%     $f(X) = g(X^{p^n})$. 
%     \item (All Roots Repeated) There exists a $K$-extension in which 
%     $f$ splits and all its roots are repeated. 
%   \end{enumerate}
% \end{prop}
% \begin{proof}
%   ($1 \implies 2$) Suppose $(f , f^\prime) = 1$.
%   Then by the Euclidean algorithm there exists
%   $\la, \mu \in K[X]$ such that $\la f + \mu f^\prime = 1$.
%   Let $K \to L$ be an extension where $f$ has a repeated root $a$.
%   By factoring $f(X) = (X - a)^2 g(X)$ in $L[X]$ and the product rule
%   for formal differentiation (which can be proved by induction),
%   we see a contradiction \[
%     1 = \la(a) f(a) + \mu(a) f^\prime(a) = 0 + 0 = 0
%   \]

%   ($2 \implies 3$) 
%   If $f^\prime \neq 0$ then since $(f , f^\prime) \neq 1$
%   and $f$ is irreducible, we must have that
%   $f$ divides $f^\prime$.
%   But then $\deg f \leq \deg f^\prime < \deg f$
%   so we must have $f^\prime = 0$.

%   ($3 \implies 4$)
%   We do induction on $\deg f$.
%   By rescaling, WLOG assume $f$ is monic.
%   Then looking at the leading coefficient of $f^\prime$,
%   we see that $\deg f = 0$ as elements inside $K$.
%   Since $\deg f > 0$ we have that $\Char{K} = p > 0$ for some prime $p$.
%   Writing $f(X) = \sum_{k \geq 0} f_k X^k$
%   and looking at $f^\prime = 0$,
%   we see that for every $f_k \neq 0$ we must have $k = 0$ in $K$.
%   In other words $k = p k_p$ for some $k_p \in \bN$.
%   Letting $g(X) = \sum_{k \geq 0} f_k X^{k_p}$ we get
%   $f(X) = g(X^p)$.
%   If $g$ is reducible then $f$ is reducible 
%   so we must have $g$ is irreducible.
%   If $g$ is separable, we are done.
%   If $g$ is inseparable,
%   then using $(1 \implies 3)$ we have $g^\prime = 0$.
%   Since $\deg g < \deg f$ we are done by induction.

%   ($4 \implies 5$)
%   Let $K \to L$ with $g(X) = (X - a_1) \dots (X - a_d)$ in $L[X]$.
%   We can extend $L$ further to split the polynomials
%   $X^{p^n} - a_i$.
%   Let $b_i \in L$ such that $b_i^{p^n} = a_i$.
%   Then by Freshman's dream 
%   \[
%     f(X) = (X^{p^n} - a_1) \dots (X^{p^n} - a_d)
%     = (X^{p^n} - b_1^{p^n}) \dots (X^{p^n} - b_d^{p^n})
%     = (X - b_1)^{p^n} \dots (X - b_d)^{p^n}
%   \]

%   ($5 \implies 1$) clear.
% \end{proof}

\section{Galois extensions and the correspondence}

\begin{dfn}
  An extension $K \to L$ is called finite Galois
  when there exists a finite subgroup $G \subs \AUT_K L$
  such that $K = L^G$.
\end{dfn}

The following is arguably the fundamental theorem of Galois theory.

\begin{prop}[Artin's characterization of finite Galois extensions]
  Let $K \to L$ be an extension.
  Then $K \to L$ is finite, normal, separable
  iff $K \to L$ is finite Galois.
  In this case the finite subgroup $G \subs \AUT_K L$ such that $K = L^G$
  must be $\AUT_K L$.
\end{prop}

% To give an idea of the proof, let us look at an example first.

% \begin{eg}
  
%   Let $K = \bQ(\om)$ where $\om^2 + \om + 1 = 0$
%   and consider the polynomial $f(T) = T^3 - 2 \in K[T]$.
%   This is irreducible because it's a cubic and has no roots in $\bQ(\om)$
%   due to $[\bQ(\om) : \bQ] = 2$.
%   Let $L := K[T] / (T^3 - 2)$ and $\al :=$ the image of $T$ in $L$.


% \end{eg}

\begin{proof}
  Slogan : \emph{set of Galois conjugates = orbit}.

  ($1 \implies 2$)
  By the embedding theorem, $\abs{\AUT_K L} \leq [L : K]$.
  We claim that $G := \AUT_K L$ works.
  Let $a \in L^G$. Goal : $a \in K$.
  It suffices to show $\min(a , K)$ is linear.
  Since $K \to L$ is normal, $\min(a, K)$ splits in $L$.
  Since $K \to L$ is separable, it suffices to show that 
  for any Galois $K$-conjugate $\al$ of $a$ we have $\al = a$.
  Let $\al \in L$ with $\min(a , K)(\al) = 0$.
  Since $a \in L^G$ is suffices to give $\si \in \AUT_K L$
  which $\si(a) = \al$.
  By the embedding theorem applied to $K(a) \to L$,
  we can extend $K(a) \simeq K(\al) \to L$ to an automorphism
  $\si : L \to L$ preserving $K$.
  This maps $a$ to $\al$ as desired.

  ($2 \implies 1$) Let $G$ be a finite subgroup of $\AUT_K L$
  such that $K = L^G$.
  For $a \in L$ we claim that \[
    \min(a , K)(T) = \prod_{\al \in G a} (T - \al) \in L[T]
  \]
  where $G a$ denotes the $G$-orbit of $a$.
  This proves that $L/K$ is normal and separable.
  Let $f \in L[T]$ be the above product.
  The claim is equivalent to showing $f \in L^G[T] = K[T]$
  and $f$ is irreducible in $K[T]$.
  Let $\si \in G$. Then \[
    \si f (T) = \si \prod_{\al \in G a} (T - \al)
    = \prod_{\al \in G a} (T - \si(\al))
    = \prod_{\tilde{\al} \in G a} (T - \tilde{\al})
    = f(T)
  \]
  Therefore $f \in K[T]$.
  For irreducibility, if $f = g h$ is a non-trivial factoring in $K[T]$
  then one of $g$ or $h$ has $a$ as a root.
  Say it's $g$, then by applying $\si \in G$ to the equation $0 = g(a)$
  we get that $g$ has all $\al \in G a$ as roots,
  i.e. $f$ divides $g$, a contradiction.

  Now we show $L / K$ is finite.
  We are expecting $G = \AUT_K L$ which should have size $[L : K]$.
  So we will bound $[L : K] \leq \abs{G}$.
  Magic claim : 
  $\dim_K L = \dim_L L[G] = \abs{G}$
  where $L[G]$ is the set of functions from $G$ to $L$.
  It will suffice for us to show that
  any $K$-linearly independent set gives rise to
  a $L$-linearly independent set in $L[G]$ with the same cardinality.
  Let $A \subseteq L$ be a finite $K$-linearly independent set.
  Define $\tilde{A} := \set{\ev_a}_{a \in A} \subs L[G]$.
  Then $\ev_\_ : A \to \tilde{A}$ is a bijection
  because $\ev_{a} = \ev_{a_1}$ implies
  $a = \ev_a (e) = \ev_{a_1}(e) = a_1$ and surjectivity is by definition.
  Claim : 
  $\tilde{A}$ is a $L$-linearly independent set in $L[G]$.
  We induct on $\abs{A}$.
  Let $\sum_{x \in X_0} \la_x ev_x = 0$ with $\la_x \in L$. 
  Suppose for a contradiction 
  that there exists $a_0 \in A$ such that $\la_{a_0} \neq 0$. 
  It suffices to show for all $a \in A$
  we have $\la_a \in L^G = K$,
  for then by evaluating at $e \in G$ gives 
  $0 = \sum_{a \in A} \la_a a$,
  implying all $\la_a = 0$. 
  So let $\sigma \in G$ with the goal of showing $\si(\la_a) = \la_a$ for
  all $a \in A$.
  By rescaling, WLOG $\la_{a_0} = 1$. 
  By induction it suffices to show \[
    \sum_{x \in X_0 \setminus\{x_0\}} (\la_x - \sigma(\la_x)) ev_x
    = 0 \in L[G]
  \]
  Let $\rho \in G$. Then we have as desired
  \begin{align*}
    \sum_{a \in A \setminus\{a_0\}} (\la_a - \sigma(\la_a)) ev_a (\rho)
    &= \sum_{x \in X_0} \la_x ev_x (\rho) - 
    \sum_{a \in A} \si(\la_a) \rho(a)
    \\
    = - \sigma \left( 
        \sum_{a \in A} \la_a \si^{-1}\rho(a)
      \right)
    &= - \si\left(
      \left(
        \sum_{a \in A} \la_a \ev_a
      \right) \si^{-1}\rho
    \right)
    = 0 
  \end{align*}
\end{proof}

\begin{prop}[The Galois correspondence]

  Let $K \to L$ be a Galois extension of fields
  and let $G := \AUT_K L$.
  Then we have an order reversing bijection \begin{cd}
    {\set{\text{$K$-subextensions }E\subseteq L}} & 
    {\set{\text{subgroups of }\mathrm{Aut}_K L}}
    \arrow["{\mathrm{Aut}_\_ L}", shift left=3, from=1-1, to=1-2]
    \arrow["\simeq"{description}, draw=none, from=1-1, to=1-2]
    \arrow["{L^\_}", shift left=3, from=1-2, to=1-1]
  \end{cd}
  Furthermore, for $E \subs L$ a $K$-subextension 
  we have the following : 
  \begin{enumerate}
    \item (Degree equals Index)
    $[E : K] = [\AUT_{K}{L} : \AUT_{E}{L}]$.
    \item (Group Action)
    For all $\sigma \in \AUT_{K}{L}$, 
    $\AUT_{\sigma E}{L} = \sigma \AUT_{E}{L} \sigma^{-1}$. 
    \item (Normality) 
    $E$ is a normal $K$-extension if and only if 
    $\AUT_{E}{L}$ is a normal subgroup of $\AUT_{K}{L}$. 
    In this case, we have the isomorphism 
    $\AUT_{K}{E} \cong \AUT_{K}{L} / \AUT_{E}{L}$. 
  \end{enumerate}
\end{prop}
\begin{proof}
  We need a lemma. 
  \begin{lem}
    Let $K \to E \to L$ be a sequence of extensions.
    \begin{enumerate}
      \item If $K \to L$ is finite normal, then $E \to L$ is finite normal.
      \item If $K \to L$ is finite separable, 
      then $E \to L$ is finite separable.
    \end{enumerate}
    \begin{proof1}
      Exercise.
    \end{proof1}
  \end{lem}
  (Surjectivity) 
  Let $H \subs \AUT_K L$ be a subgroup.
  Then $\AUT_{L^H} L = H$ by the characterisation of Galois extensions.
  Now let $E \subs L$ be a $K$-subextension.
  Then by the above lemma, $L/E$ is Galois so $E = L^{\AUT_E L}$.

  (Injectivity) This actually does not use any Galois theory
  and is true for any partially ordered set.
  Here is the statement.
  \begin{lem}
    Let $I, J$ be partially ordered sets,
    $F : I \to J$ and $G : J \to I$ be order reversing functions
    satisfying: 
    \begin{itemize}
      \item (Adjunction) For all $x \in I$ and $y \in J$,
      $x \leq G(y)$ iff $y \leq F(x)$.
    \end{itemize}
    Then $FGF = F$ and $GFG = G$.
    In particular, $F$ and $G$ induce a bijection on
    the images $FI, GJ$.
    \begin{proof1}
      Exercise.
    \end{proof1}
  \end{lem}
  (Degree equals index) Use the above lemma and 
  the characterisation of Galois extensions.
  
  (Group action) Exercise.

  (Normality)
  If $E / K$ is normal,
  then image-invariance of normal extensions we get a
  well-defined morphism of groups by restriction \[
    \AUT_K L \to \AUT_K E
  \]
  The kernel is by definition $\AUT_E L$ so it is normal.

  If $\AUT_E L$ is normal,
  then for any $\si \in \AUT_K L$ we have 
  \[
    \si E = L^{\AUT_{\si E} L}
    = L^{\si \AUT_E L \si^{-1}}
    = L^{\AUT_E L} = E
  \]
  so restriction gives a well-defined morphism of groups
  $\AUT_K L \to \AUT_K E$.
  Let $G$ be the image.
  Then $E^G = E \cap L^{\AUT_K L} = E \cap L^G = E \cap K = K$
  so $E / K$ is Galois and hence normal.
  By the characterisation of Galois extensions,
  $G$ must be all of $\AUT_K E$ and hence 
  by the first isomorphism theorem of groups 
  we have $\AUT_K E \simeq \AUT_K L / \AUT_E L$.

\end{proof}

\begin{eg}
  
  Let us compute the Galois group of 
  $T^4 - a \in \bQ[T]$ over $K = \bQ$ where 
  $a$ is a positive integer with no square factors.
  
  Let $p > 0$ be a prime that divides $a$.
  Then $T^4 - a$ satisfies Eisenstein's criterion and hence is
  irreducible in $\bQ[T]$.
  Let $L / K$ be a splitting field of $T^4 - a$
  and $\al \in L$ any root.
  This is a Galois extension because $\bQ$ is characteristic zero.
  By separability of $T^4 - a$,
  there exists another root $\be$ not equal to $\pm \al$.
  Let $i := \be / \al$.
  Then $0 = i^4 - 1 = (i - 1)(i + 1)
  (i^2 + 1)$
  implies $i^2 + 1 = 0$.
  So the four roots are $\al , \al i , \al i^2 , \al i^3$.

  Let $\sqrt[4]{2} \in \bR$ be the unique positive fourth-root of $2$.
  Using the embedding theorem,
  there exists an embedding $\phi : L \to \bC$
  such that $\phi(\al) = \sqrt[4]{2}$.
  From this, we deduce $i \notin \bQ(\al)$ because
  if it were it would give an element $\phi(i) \in \phi \bQ(\al) \subs \bR$
  which is not fixed by complex conjugation.
  It follows that $T^2 + 1$ is irreducible in $\bQ(\al)[T]$
  and hence $[L : \bQ] = [L : \bQ(\al)] [\bQ(\al) : \bQ] = 2 \cdot 4 = 8$.
  \footnote{
    There should be a way to do this without using $\bR$ but
    this is probably the easiest way.
  }

  Using embedding theorem for $L / \bQ(\al)$,
  we get $\tau \in \AUT_\bQ L$ such that \begin{align*}
    \tau(i) = -i && \tau(\al) = \al
  \end{align*}
  Since $\deg \min(\al , \bQ(i)) = [L : \bQ(i)] = 4$ by the tower law,
  we have $\min(\al , \bQ(i)) = T^4 - 2$.
  Using embedding theorem again, 
  we have $\si \in \AUT_\bQ L$ such that \begin{align*}
    \si(i) = i && \si(\al) = \al i
  \end{align*}
  We have $\si^k(\al) = \al i^k$ so $\si$ has order 4.
  \begin{align*}
    \tau \si \tau^{-1} (i) &= \tau \si (-i) = \tau(- i) = i \\
    \tau \si \tau^{-1} (\al) &= \tau \si (\al) = 
    \tau(\al i) = - \al i = \si^{-1}(\al)
  \end{align*}
  So $\tau \si \tau^{-1} = \si^{-1}$ and thus 
  $\AUT_\bQ L \simeq D_8$.
  We have the following classification of subgroups of $D_8$
  \begin{cd}
    && {\langle 1 \rangle} \\
    {\langle \tau \rangle} & {\langle \tau \sigma^2 \rangle} & {\langle \sigma^2 \rangle} & {\langle \tau \sigma \rangle} & {\langle \tau \sigma^3 \rangle} \\
    & {\langle \tau , \sigma^2 \rangle} & {\langle \sigma \rangle} & {\langle \tau \sigma , \sigma^2 \rangle} \\
    && {\langle \sigma , \tau\rangle}
    \arrow[from=1-3, to=2-1]
    \arrow[from=1-3, to=2-2]
    \arrow[from=1-3, to=2-3]
    \arrow[from=1-3, to=2-4]
    \arrow[from=1-3, to=2-5]
    \arrow[from=2-1, to=3-2]
    \arrow[from=2-2, to=3-2]
    \arrow[from=2-3, to=3-2]
    \arrow[from=2-3, to=3-3]
    \arrow[from=2-3, to=3-4]
    \arrow[from=2-4, to=3-4]
    \arrow[from=2-5, to=3-4]
    \arrow[from=3-2, to=4-3]
    \arrow[from=3-3, to=4-3]
    \arrow[from=3-4, to=4-3]
  \end{cd}
  The corresponding intermediate extensions are : 
  \begin{cd}
    && {\mathbb{Q}(\alpha, i)} \\
    {\mathbb{Q}(\alpha)} & {\mathbb{Q}(\alpha i)} & {\mathbb{Q}(i , \alpha^2)} & {\mathbb{Q}(\alpha - i \alpha)} & {\mathbb{Q}(\alpha + i \alpha)} \\
    & {\mathbb{Q}(\alpha^2)} & {\mathbb{Q}(i)} & {\mathbb{Q}(i \alpha^2)} \\
    && {\mathbb{Q}}
    \arrow[from=2-1, to=1-3]
    \arrow[from=2-2, to=1-3]
    \arrow[from=2-3, to=1-3]
    \arrow[from=2-4, to=1-3]
    \arrow[from=2-5, to=1-3]
    \arrow[from=3-2, to=2-1]
    \arrow[from=3-2, to=2-2]
    \arrow[from=3-2, to=2-3]
    \arrow[from=3-3, to=2-3]
    \arrow[from=3-4, to=2-3]
    \arrow[from=3-4, to=2-4]
    \arrow[from=3-4, to=2-5]
    \arrow[from=4-3, to=3-2]
    \arrow[from=4-3, to=3-3]
    \arrow[from=4-3, to=3-4]
  \end{cd}
  To compute fixed subfields $L^H$ of a given subgroup $H$ of $\AUT_\bQ L$,
  one can use linear algebra:
  Choose a $\bQ$-basis for $L$,
  for each $\si \in H$
  write the matrix $A$ given by the $K$-linear map $x \mapsto \si(x)$
  and compute the kernel of $A - I$ where $I$ is the identity matrix.
  Alternatively, one can check that it is invariant and then check the degree.
  For example, $\tau\si(\al - i \al) = \tau(\al i + \al) = \al - \al i$
  so $\bQ(\al - i \al) \subs L^{\< \tau \si \>}$.
  \[
    (\al - i \al)^4 = (- 2 i \al^2)^2 = - 4 a
  \]
  so $[\bQ(\al - i \al) : \bQ] \leq 4$.
  On the other hand, if $i \in \bQ(\al - i \al)$ then
  $\al = (\al - i \al + i (\al - i \al)) / 2$ implies
  $L = \bQ(\al - i \al)$ which would imply $8 = [L : \bQ] \leq 4$
  a contradiction.
  Therefore $[L : \bQ(\al - i \al)] = 2$ and hence
  $[\bQ(\al - i \al) : \bQ] = 4$.
  Since $[L^{\< \tau \si \>} : \bQ] = [\< \si , \tau \> : \< \tau \si\>] = 4$
  we conclude $\bQ(\al - i \al) = L^{\< \tau \si \>}$.
\end{eg}

\section{Cyclotomic extensions, Cyclic extensions}

We saw in the example of $T^3 - 2$ that
in understanding its roots, the roots of $T^3 - 1$ appeared.
This reduces the understanding of radical Galois extensions
into two steps : \emph{cyclotomic} and \emph{cyclic} extensions.
We study these as stepping stones towards understanding 
radical extensions.

\begin{dfn}
  
  Let $K$ be a field and $f \in K[T]$.
  A \emph{splitting field of $f$} is an extension
  $K \to L$ that splits $f$ and is generated by the roots of $f$.
\end{dfn}
It will be useful to have another characterisation of Galois extensions.

\begin{prop}[Splitting field characterisation of Galois extensions]
  
  Let $K \to L$ be an extension.
  Then $L / K$ is the splitting field of a separable $f \in K[T]$
  iff $L / K$ is Galois.
\end{prop}
\begin{proof}
  ($\implies$)
  Let $G = \AUT_K L$ which is finite by the embedding theorem.
  We know that $L / L^G$ is Galois so STS $L^G = K$.
  Since $f$ is separable,
  $\min(\al , K)$ is also separable for any root $\al$ of $f$.
  Since the roots of $f$ generate $L$ over $K$,
  by the embedding theorem we have $\abs{G} = [L : K]$.
  Then $[L^G : K] = [L : K] / [L : L^G] = [L : K] / [L : K] = 1$.

  ($\limplies$)
  By the characterisation of normal extensions,
  $L / K$ is the splitting field of some $f \in K[T]$.
  Remove all repeated irreducible factors of $f$ so that $f$ is square-free.
  Any pair of distinct irreducible factors $g , h$ of $f$ 
  must satisfy $1 = \la g + \mu h$ for some $\la , \mu \in K[T]$.
  It follows that they do not share roots in any extension of $K$.
  The irreducible factors of $f$ are (scalar multiplies of) minimal polynomials,
  which are separable because $L / K$ is Galois.
  Thus $f$ is a separable polynomial.
\end{proof}

\begin{prop}[Galois groups of cyclotomic extensions]
  
  Let $n \in \bN$ and $L / K$ the splitting field of $X^n - 1 \in K[T]$.
  Assume that $X^n - 1$ is separable,
  or equivalently $n \neq 0$ as elements of $K$.
  Let $\mu_n \subs L^\times$ be the subgroup of roots of $X^n - 1$.
  A \emph{primitive $n$-th root of unity} is defined as a generator of $\mu_n$.
  Then 
  \begin{enumerate}
    \item there are $\phi(n)$ many primitive $n$-th roots of unity in $L$
    \item the group morphism \begin{align*}
      (\bZ / n \bZ)^\times &\to \AUT_\GRP \mu_n \\
      k &\mapsto (z \mapsto z^k)
    \end{align*}
    is an isomorphism,
    where $\AUT_{\GRP} \mu_n$ denotes the group of group automorphisms
    of $\mu_n$.
    The restriction $\GAL(L / K) \to \AUT_{\GRP} \mu_n$ is injective,
    so $\GAL(L / K)$ is abelian.
  \end{enumerate}
\end{prop}
\begin{proof}
  (1) Let $\mu_n^d \subs \mu_n$ be the subset of elements with order $d$.
  Then by strong induction on $n$ we have\footnote{
    Here is a proof of $n = \sum_{0 \leq d \vert n} \phi(d)$.
    We take as definition $\phi(d) := \abs{(\bZ / d \bZ)^\times}$.
    Then the chinese remainder theorem implies $\phi$ is multiplicative
    so it suffices to prove the result for $n = p^a$ where $p > 0$ is prime
    and $a > 0$.
    Now $p^a = (p^a - p^{a - 1}) + \cdots + (p - 1) + 1
    = \phi(p^a) + \phi(p^{a-1}) + \cdots + \phi(p) + 1$
    because an element in $\bZ / p^k \bZ$ is invertible
    iff it is invertible mod $p$.
  } \[
    \abs{\mu_n^n} = \abs{\mu_n} - \sum_{n > d \vert n} \abs{\mu_n^d}
    n - \sum_{n > d \vert n} \phi(d) = \phi(n)
  \]

  (2) Being a splitting field of a separable polynomial,
  it makes sense to talk about the Galois group $L / K$.
  We give an inverse group morphism.
  Since $\phi(n) > 0$ there exists $z_0 \in \mu_n$ with order $n$.
  For any $\si \in \AUT_{\GRP} \mu_n$,
  there is a unique $k_\si \in \bZ / n\bZ$ such that
  $\si(z_0) = z_0^{k_\si}$.
  Since $\si$ has to send $z_0$ to another element of order $n$,
  we must have $(n , k_\si) = 1$ i.e. $k_\si \in (\bZ / n \bZ)^\times$.
  Then $\si \mapsto k_\si$ gives the desired inverse.\footnote{
    Although the definition of the inverse used a choice of 
    generator of $\mu_n$, it is independent of this choice because
    inverse of group morphisms are unique and
    $(\bZ / n \bZ)^\times \to \AUT_{\GRP} \mu_n$ does not
    use any choices of generator of $\mu_n$.
  }
\end{proof}

\begin{rmk}
  It is possible to show that when $K = \bQ$,
  the morphism $\GAL(\bQ(\mu_n) / \bQ) \to \AUT_{\GRP} \mu_n$ is surjective
  and hence bijective.
  This is not necessary for solvability of polynomials
  so we will return to this later.
\end{rmk}

\begin{prop}[Characterization of cyclic extensions]
  
  Let $n \in \bZ_{> 0}$ and 
  $K \to L$ be an extension where $T^n - 1 \in K[T]$ is split 
  and separable in $K$.
  Then \begin{enumerate}
    \item if $L = K(\al)$ where $\al^n \in K$
    and is the minimal power of $\al$ in $K$, 
    then $L / K$ is Galois 
    and the map \begin{align*}
      \GAL(L / K) \to \mu_n \\
      \si \mapsto \si(\al) / \al
    \end{align*}
    is group isomorphism.
    Hence $\GAL(L / K)$ is cyclic.
    \item Conversely if $L / K$ is Galois with $\GAL(L / K)$ cyclic order $n$
    then there exists $\al \in L$ such that $L = K(\al)$ and $\al^n$ is the
    minimal power of $\al$ in $K$.
  \end{enumerate}
\end{prop}
This is completely analogous to the situation 
$\bQ(\sqrt[3]{2} , \om) / \bQ(\om)$.
\begin{proof}
  (1) (Galois)
  $L / K$ is the splitting field of $T^n - \al^n$
  which is separable by separability of $T^n - 1$.

  (Group morphism)
  For $\si , \rho \in \GAL(L / K)$
  we have \[
    \frac{\si(\rho(\al))}{\al} 
    = \frac{\si(\rho(\al))}{\rho(\al)} \frac{\rho(\al)}{\al}
    = \frac{\si(\al)}{\al} \frac{\rho(\al)}{\al}
  \]
  because $\rho(\al) = \al z$ for some $z \in \mu_n$ so \[
    \frac{\si(\rho(\al))}{\rho(\al)}
    = \frac{\si(\al) z}{\al z} = \frac{\si(\al)}{\al}
  \]

  (Bijective)
  Injectivity follows from kernel being trivial
  because any $\si$ is determined by what it does on $\al$.
  Suppose for a contradiction that $\GAL(L / K) \to \mu_n$ is not surjective.
  Then the image of $\GAL(L / K)$ is a subgroup of order $d < n$
  so by Lagrange's theorem 
  for all $\si \in \GAL(L / K)$ we have $(\si(\al) / \al)^d = 1$.
  This says $\si(\al^d) = \al^d$ i.e. $\al^d \in L^G = K$
  which contradicts minimality of $n$.

  (2) Let $\si \in \GAL(L / K)$ be a generator.
  The proof of (1) shows that we are expecting
  $\al \in L$ to be such that $\si(\al) / \al \in \mu_n$,
  i.e. $\al$ is an eigenvector of $\si$ with eigenvalue $z \in \mu_n$.
  So consider $\si$ as a $K$-linear map $L \to L$.
  Then the minimal polynomial of $\si$ divides $T^n - 1$ in $K[T]$.
  This is split and separable over $K$ so 
  the minimal polynomial of $\si$ is split and separable over $K$.
  This occurs iff $\si$ is diagonalizable as a $K$-linear map.
  The eigenvalues of $\si$ are precisely the roots of its minimal polynomial,
  which divides $T^n - 1$ so
  consequently there exists $\al \in L$ with
  eigenvalue $z \in \mu_n$, i.e. $\si(\al) = z \al$.
  Then $\si((\al)^n) = (\si(\al))^n = (z \al)^n = \al^n$ so $\al^n \in L^G = K$.
  Let $\tilde{n}$ be the minimal power of $\al$ in $K$.
  Then by $(1)$ we have $\tilde{n} = \abs{\GAL(L / K)} = n$.
\end{proof}

\section{Radical extensions}

Today we discuss solvability polynomials.

\begin{dfn}

  Let $K \to L$ be an extension.
  We say it is \emph{radical} when
  there exists a chain of subextensions 
  \[
    K = L_0 \to L_1 \to \cdots \to L_{n-1} \to L_n = L
  \]
  such that each $L_{i+1} = L_i(\al_i)$
  for some $\al_i$ with $\al_i^{d_i} \in L_i$ for some $d_i > 0$.

  For $f \in K[T]$ we say $f$ is \emph{solvable by radicals} when
  there exists a radical extension $L / K$ which splits $f$.
\end{dfn}

Some group theoretic things we need... 

\begin{dfn}
  
  Let $G$ be a finite group.
  Then $G$ is called solvable when
  there exists a chain \[
    1 = H_0 \triangleleft H_1 \triangleleft
    \cdots \triangleleft H_{n-1} \triangleleft H_n = G
  \]
  such that $H_{n+1} / H_n$ is cyclic.
\end{dfn}

\begin{prop}
  
  Suppose we have a normal subgroup $N$ of a finite group G.
  \[
    1 \to N \to G \to G / N \to 1
  \]
  If $N$ and $G / N$ are solvable, then $G$ is solvable.
  If $G$ is solvable, then $N$ is solvable.
  \footnote{
    One can prove in this case that $G / N$ is solvable, too,
    but this is not relevant for solvability of polynomials.
  }
\end{prop}
\begin{proof}
  Exercise in group theory.
\end{proof}

The main result is : 
\begin{prop}[Characterization of solvable polynomials in characteristic zero]
  
  Let $K$ be a characteristic zero field and $f \in K[T]$ be irreducible.
  Then $f$ is solvable by radicals iff 
  there exists a splitting field $L / K$ of $f$
  such that $\GAL(L / K)$ is solvable.
\end{prop}

\begin{rmk}
  In the above,
  $L / K$ is normal by the characterization of finite normal extensions.
  $K$ characteristic zero implies all extensions of $K$ are separable, 
  so $L / K$ is indeed Galois and it makes sense to talk about its Galois group.
  Furthemore if $\tilde{L} / K$ is another splitting field of $f$ then
  by the embedding theorem there exists an isomorphism $\ga : L \simeq \tilde{L}$
  of extensions of $K$.
  It follows that $\ga \_ \ga^{-1} : \GAL(L / K) \to \GAL(\tilde{L} / K)$
  is an isomorphism of groups.\footnote{
    This is analogous to the following phenomenon from algebraic topology : 
    given a topological space $X$ and a path $\ga$ from
    a point $x$ to $\tilde{x}$,
    then $\ga \_ \ga^{-1}$ gives an isomorphism 
    $\pi_1(X , x) \simeq \pi_1(X , \tilde{x})$.
    These two are united in \emph{algebraic geometry}.
  }
  So for a polynomial $f$ solvable by radicals,
  \emph{all} splitting fields of $f$ have solvable Galois groups.
\end{rmk}

\begin{proof}[Proof of characterization of solvable polynomials in characteristic zero]
  
  $(\implies)$ Assume there is a tower of simple radical extensions 
  \[
    K = L_0 \to L_1 \to \cdots \to L_{n-1} \to L_n = L
  \]
  where $L_{i + 1} = L_i(\al_i)$ for some $\al_i^{d_i} \in L_i$ and $d_i > 0$,
  and that $L$ contains a splitting field of $f$.
  \textbf{Let us first assume $L / K$ is Galois.}
  Then $L / K$ Galois implies it splits $X^N - 1$ where $N = d_1 \cdots d_n$.
  It is separable by the assumption that $K$ is characteristic zero.
  Let $\tilde{L}_0 := L_0(\mu_N)$ and
  $\tilde{L}_{i+1} := \tilde{L}_i(\al_i)$.
  \begin{cd}
    {L_0} & {L_1} & \cdots & {L_{n - 1}} & {L_n = L} \\
    {\tilde{L}_0} & {\tilde{L}_1} & \cdots & {\tilde{L}_{n - 1}} & {\tilde{L}_{n} = L}
    \arrow[from=1-1, to=1-2]
    \arrow[from=1-1, to=2-1]
    \arrow[from=1-2, to=1-3]
    \arrow[from=1-2, to=2-2]
    \arrow[from=1-3, to=1-4]
    \arrow[from=1-3, to=2-3]
    \arrow[from=1-4, to=1-5]
    \arrow[from=1-4, to=2-4]
    \arrow["{=}", from=1-5, to=2-5]
    \arrow[from=2-1, to=2-2]
    \arrow[from=2-2, to=2-3]
    \arrow[from=2-3, to=2-4]
    \arrow[from=2-4, to=2-5]
  \end{cd}
  Applying the main theorem of Galois theory we obtain 
  a sequence of subgroups \[
    \GAL(\tilde{L}_n / \tilde{L}_n) 
    \subs \GAL(\tilde{L}_n / \tilde{L}_{n-1})
    \subs \cdots
    \subs \GAL(\tilde{L}_n / \tilde{L}_1)
    \subs \GAL(\tilde{L}_n / \tilde{L}_0)
    \subs \GAL(\tilde{L}_n / L_0) = \GAL(L / K) 
  \]
  Then
  \begin{enumerate}
    \item each factor group 
    $\GAL(\tilde{L}_n / \tilde{L}_{i}) / \GAL(\tilde{L}_n / \tilde{L}_{i})
    \simeq \GAL(\tilde{L}_{i+1} / \tilde{L}_{i})$ is cyclic
    by the characterisation of cyclic extensions.
    \item For the final factor group at the top,
    $\tilde{L}_0 / L_0$ is a cyclotomic extension.
    So it has abelian Galois group,
    which is in particular solvable
    by, say, the classification of finite abelian groups.
  \end{enumerate}
  Thus $\GAL(L / K)$ is solvable.

  To complete the proof of the forward direction,
  we need to show that we can always enlarge $L$ so that
  $L / K$ is not just radical but also Galois.
  By splitting minimal polynomials of generators of $L / K$,
  we can find $N / L$ such that $N / K$ is finite normal.
  Since $K$ is characteristic zero, $N / K$ is separable and hence Galois.
  But $N$ is made with choices (the generators of $L / K$)
  so we do not know immediately that $N / K$ is radical.

  Let $\GAL(N / K) = \set{\si_1 , \dots , \si_{[N : K]}}$
  with $\si_1 = e$.
  The reason why $L / K$ is not Galois is more or less
  because we don't have the Galois conjugates of $\al_i$.
  So we add them in.
  Define the tower of subextensions \begin{align*}
    K &= L_{1 , 0} \subs L_{1 , 1} \subs \cdots \subs L_{1, n-1} \subs L_{1 , n} \\
    &= L_{2 , 0} \subs L_{2 , 1} \subs \cdots \subs L_{2 , n-1} \subs L_{2 , n} \\
    &= L_{3 , 0} \subs \cdots  \\
    &= L_{[N : K] , 0} \subs L_{[N : K] , 1} \subs \cdots \subs L_{[N : K] , n} =: M
  \end{align*}
  where $L_{i , j + 1} = L_{i , j}(\si_i(\al_j))$.
  Goal : each step is simple radical and $M / K$ is Galois.
  The point is that 
  \begin{itemize}
    \item $\si_2 L_{1 , 1} = \si_2 L_{1 , 0}(\al_1)
  = L_{1 , 0}(\si_2(\al_1)) \subs L_{2 , 0}(\si_2(\al_1)) = L_{2 , 1}$
  \item $\si_2 L_{1 , 2} = \si_2 L_{1, 1}(\al_2)
  \subs L_{2 , 1}(\si_2(\al_2)) = L_{2 , 2}$
  \item by induction the same for the entirety of second row.
  \item By the same reasoning, we get for every $i$-th row
  $\si_i L_{1, j} \subs L_{i , j}$ for all $j$.
  \end{itemize}
  From this we get 
  \[
    (\si_i(\al_j))^{d_j} = \si_i(\al_j^{d_j}) \in L_{i, j}
  \]
  so that $L_{i , j + 1} / L_{i , j}$ is simple radical.
  To show $M / K$ is Galois, it suffices by the characterisation of
  Galois extensions to show that $M$ is stable under the action of $\GAL(N / K)$.
  For this we guess another construction of $M$.
  From the proof of the Tower law,
  we define $\tilde{M}$ as the set of finite $K$-linear combinations
  of $\si_1(x_1) \cdots \si_{[N : K]}(x_{[N : K]})$ where $x_i \in L$.
  This is a subring of $N$ containing $K$
  and finiteness of $L / K$ implies finiteness of $\tilde{M}$
  as a $K$-vector space.
  It follows that $\tilde{M}$ is a $K$-subextension of $N / K$.
  By looking at the proof of the Tower law, $M \subs \tilde{M}$.
  Conversely, any $\si_1(x_1) \cdots \si_{[N : K]}(x_{[N : K]})
  \in (\si_1 L) \cdots (\si_{[N : K]} L) 
  \subs L_{1 , n} \cdots L_{[N : K] , n} \subs M$
  so $\tilde{M} \subs M$ and hence $M = \tilde{M}$.
  \footnote{
    The trick of constructing $\tilde{M}$ here is called 
    taking \emph{normal closure}.
    It comes from trying to force the image invariance property 
    in the characterisation of finite normal extensions.
  }

  ($\limplies$)
  Suppose $L / K$ is a splitting field of $f$ and
  $\GAL(L / K)$ is solvable.
  Again, for the characterisation of cyclic extensions to apply
  we need enough roots of unity in our base field.
  Let $L \to \tilde{L}$ be a splitting field of $T^{[L : K]} - 1 \in L[T]$
  and $\tilde{K} := K(\mu_{[L : K]}) \subs \tilde{L}$.
  The extension $\tilde{L} / K$ is the splitting field of
  $(T^{[L : K]} - 1) f \in K[T]$, and hence Galois because
  we are in characteristic zero.
  By the main theorem of Galois theory, we have \[
    \GAL(\tilde{L} / K) / \GAL(\tilde{L} / L) \simeq \GAL(L / K)
  \]
  The latter is solvable and the kernel is solvable too 
  because $\tilde{L} / L$ is a cyclotomic extension.
  Thus $\GAL{\tilde{L} / K}$ is also solvable.
  Since $\GAL{\tilde{L} / \tilde{K}}$ is a normal subgroup,
  it is also solvable.
  So we have
  \[
    1 = H_0 \triangleleft H_1 \triangleleft
    \cdots \triangleleft H_{n-1} \triangleleft H_n = \GAL(\tilde{L} / \tilde{K})
  \]
  with cyclic factor groups.
  To apply the characterisation of cyclic extensions
  to get $\tilde{K} \to \tilde{L}$,
  we need to know $\abs{H_{i + 1} / H_i}$ divide $[L : K]$
  so that we have the correct roots of unity.
  $\abs{H_{i + 1} / H_i}$ divides $\abs{\GAL(\tilde{L} / \tilde{K})}$
  so it STS that the composition 
  \[
    \GAL{\tilde{L} / \tilde{K}} \to \GAL(\tilde{L} / K) \to \GAL{L / K}
  \]
  is injective.
  If $\si \in \GAL(\tilde{L} / \tilde{K})$ fixes $L$
  then it fixes the roots of $f$ and $T^n - 1$.
  But these generate $\tilde{L}$ over $K$ so then $\si = 1$.
  Hence, $\tilde{K} \to \tilde{L}$ is radical.
  Since $K \to \tilde{K}$ is cyclotomic and so also radical,
  we have thus that $K \to \tilde{L}$ is radical, completing the proof.
\end{proof}

\section{Finite fields, Frobenius lifts and existence of non-solvable quintic}

By the characterisation of solvability over characteristic zero,
to show that there exists quintics with roots 
\emph{inexpressible} in terms of basic arithmetic and radicals,
it suffices to give an irreducible quintic with non-solvable Galois group.
We claim that $T^5 - T - 1 \in \bQ[T]$ has Galois group $S_5$ 
which is not solvable.
To compute its Galois group, we introduce an effective technique
using \emph{finite fields} called \emph{Frobenius lifts}.

\begin{dfn}
  
  Let $A$ be a ring where $p = 0$.
  The \emph{Frobenius map} is the map $x \mapsto x^p$ on $A$.
  This is a ring morphism by freshman's dream.
\end{dfn}

\begin{prop}[Classification of finite fields]
  
  Let $p > 0$ be a prime.
  \begin{itemize}
    \item (Existence) For $n > 0$,
    let $\bF_{p^n}$ be a splitting field of $X^{p^n} - X$ over $\bF_p$.
    Then $\bF_{p^n}$ is a field with $p^n$ elements.
    This is well-defined up to isomorphism as $\bF_p$ extensions.

    \item (Uniqueness) Any finite extension $\bF_p \to F$
    must be isomorphic to $\bF_{p^n}$ for some $n > 0$
    as extensions of $\bF_p$.

    \item (Galois) $\bF_p \to \bF_{p^n}$ is finite Galois
    with Galois group cyclic order $n$ generated by 
    the Frobenius map $x \mapsto x^p$.

    More generally, for $0 < n$ and $0 \leq d$,
    the extension $\bF_{p^{n}} \to \bF_{p^{n + d}}$ is finite Galois
    with Galois group cyclic order $d$ generated by
    $x \mapsto x^{p^n}$.
  \end{itemize}
\end{prop}
\begin{proof}
  (Existence) Because the Frobenius is a ring morphism on $\bF_{p^n}$,
  the set of roots of $X^{p^n} - X$ forms a subfield containing $\bF_p$.
  It follows that this must be all of $\bF_{p^n}$.

  (Uniqueness) Let $n := [F : \bF_p]$.
  Then $\abs{F} = p^{[F : \bF_p]}$.
  By Lagrange's theorem on groups,
  we have that any $x \in F^\times$ must satisfy 
  $x^{\abs{\bF_p^\times}} - 1 = 0$.
  It follows that the elements of $F$ are precisely all the roots of 
  the polynomial $X^{p^{n}} - X$
  and hence must be a splitting field for it.

  (Galois) For $x \in \bF_{p^n}$,
  $x^p = x$ iff $x \in \bF_p$.
  The result follows from the characterisation of finite Galois extensions.
  We leave the general case as an exercise.
\end{proof}

\begin{eg}
  Let us find all the monic irreducible quadratics in $\bF_3[T]$.
  By the classification of finite fields,
  they are precisely the minimal polynomials of
  $x \in \bF_9 \setminus \bF_3$.
  This implies there are precisely three of them.
  At this point we can guess.
  The following quadratics do not have roots in $\bF_3$ and
  hence are irreducible.
  \begin{itemize}
    \item $T^2 + 1$
    \item $T^2 - T - 1$
    \item $T^2 + T - 1$
  \end{itemize}
\end{eg}

\begin{prop}[Frobenius lifts]
  
  Let $f \in \bZ[T]$ be monic and separable,
  $\bQ \to K$ a splitting field of $f$.
  Let $S \subs K$ be the set of roots of $f$
  and consider the subring generated by $S$
  \[
    A := \bZ[S] \subs K
  \]
  Let $p > 0$ be a prime such that
  the mod $p$ reduction $\bar{f} \in \bF_p[T]$ is separable.
  Then : 
  \begin{enumerate}
    \item There exists a maximal ideal $\f{q} \subs A$
    which contains $p$.
    We call $\f{q}$ a \emph{prime lying above $p$}.
    \item Let $\f{q}$ be a prime lying above $p$.
    Define the \emph{decomposition group}
    \[
      D(\f{q} / p) := \set{\si \in \GAL(K / \bQ) \st \si(\f{q}) = \f{q}}
    \]
    Then 
    \begin{enumerate}
      \item $\ka(\f{q}) := A / \f{q}$ is a splitting field of $\bar{f}$
      \item The ring morphism $\bZ[S] \to \ka(\f{q}) , x \mapsto \bar{x}$ 
      induces a bijection between $S$ and its image $\bar{S}$.
      The action of $D(\f{q} / p)$ on $S$ is compatible
      with the action of $\GAL(\ka(\f{q}) / p)$ on $\bar{S}$,
      i.e. the following diagram commutes \begin{cd}
        D(\f{q} / p) \times S & S \\
        \GAL(\ka(\f{q}) / p) \times \bar{S} & \bar{S}
        \arrow[from=1-1, to=1-2]
        \arrow[from=1-2, to=2-2, "{\sim}"]
        \arrow[from=1-1, to=2-1]
        \arrow[from=2-1, to=2-2]
      \end{cd}
  
      \item 
      The group morphism $D(\f{q} / p) \to \GAL(\ka(\f{q}) / \bF_p)$
      is bijection.
    \end{enumerate}
    In particular, there exists a unique $\phi \in \GAL(K / \bQ)$
    which such that $\res{\phi}{A} = \FROB$ mod $\f{q}$.
  \end{enumerate}
\end{prop}
Let's see the applications to Galois groups before the proof.
\begin{prop}[Dedekind's result on cycle shapes]
  
  Let $f \in \bZ[T]$ be monic separable, $K / \bQ$ a splitting field
  and $S \subs K$ the set of its roots.
  Let $p > 0$ a prime such that mod $p$ reduction $\bar{f} \in \bF_p[T]$
  is separable.
  Suppose that $\bar{f}$ factors into irreducible polynomials
  of degree $n_1 , \dots , n_r$.
  Then there exists $\si \in \GAL(K / \bQ)$
  such that under the injection $\GAL(K / \bQ) \to \AUT S$,
  $\si$ has cycle shape $(n_1) \cdots (n_r)$.
\end{prop}
\begin{proof}
  Let $\f{q}$ be as in the previous proposition and 
  $\si$ be a Frobenius lift from $\ka(\f{q})$.
  Then the cycle shape of $\si$ acting on $S$ is the cycle shape
  of the Frobenius acting on $\bar{S}$.
  The cycles are precisely the orbits under the action of
  the Galois group $\GAL(\ka(\f{q}) / p)$.
  Elements of each orbit share the same minimal polynomial,
  and the set of minimal polynomials ranging across the orbits
  is precisely the irreducible factors of $\bar{f}$
  (up to scaling by $\bF_p^\times$).
\end{proof}

\begin{prop}
  
  The polynomial $T^5 - T - 1 \in \bQ[T]$ has Galois group $S_5$
  and hence is not solvable by radicals.
\end{prop}
\begin{proof}
  Mod 2 we get $(T^2 + T + 1)(T^3 + T^2 + 1)$
  which are irreducible because they do not have roots in $\bF_2$.
  So there exists an element of the Galois group
  with cycle shape $(2) (3)$ and hence there exists one
  with cycle shape $(2)$, i.e. a transposition.

  Mod 3 it is irreducible.
  It has no roots and we can check
  it is not divisible by the three monic irreducible quadratics
  in $\bF_3[T]$.
  \begin{itemize}
    \item $T^5 - T - 1 = (T^2 + 1)(T^3 - T) - 1$
    \item $T^5 - T - 1 = (T^2 - T - 1)(T^3 + T^2 - T - 1) + 1$
    \item $T^5 - T - 1 = (T^2 + T - 1)(T^3 - T^2 - T) + (T - 1)$
  \end{itemize}
  So there exists an element of the Galois group
  with cycle shape $(5)$.
  \begin{lem}
    Let $G \subs S_p$ where $p > 0$ is prime.
    Suppose $G$ contains a tranposition and a $p$-cycle.
    Then $G = S_p$.
    \begin{proof1}
      Exercise in group theory.
    \end{proof1}
  \end{lem}
  So the Galois group is $S_5$.
  The fact that this is not solvable is a group theory fact
  so we omit it.
\end{proof}

We give another application of Frobenius lifts.
\begin{prop}[Cyclotomic extensions in characteristic zero]
  
  Let $n > 0$ and let $\bQ(\mu_n) / \bQ$ be a splitting field
  of $T^n - 1$.
  Recall we have a canonical isomorphism 
  $\AUT_{\GRP} \mu_n \simeq (\bZ / n \bZ)^\times$
  and hence an injection \[
    \GAL(\bQ(\mu_n) / \bQ) \to (\bZ / n \bZ)^\times
  \]
  This is in fact an isomorphism.
\end{prop}
\begin{proof}
  Let $m \in (\bZ / n \bZ)^\times$.
  This corresponds to $x \mapsto x^m \in \AUT_{\GRP} \mu_n$.
  To show $m$ is in the image of $\GAL(\bQ(\mu_n) / \bQ)$,
  it suffices to do case of $m = p$ a prime.
  By assumption, $n \neq 0$ mod $p$ so $T^n - 1$ is separable over $\bF_p$.
  Then we have a commuting diagram of group morphisms : 
  \begin{cd}
    {D(\mathfrak{q} / p)} & {\mathrm{Gal}(\mathbb{Q}(\mu_n) / \mathbb{Q})} & {\mathrm{Aut}_{\mathrm{Grp}} \mu_n(\mathbb{Q})} \\
    {\mathrm{Gal}(\mathbb{F}_p(\mu_n) / \mathbb{F}_p)} && {\mathrm{Aut}_{\mathrm{Grp}} \mu_n(\mathbb{F}_p)}
    \arrow[from=1-1, to=1-2]
    \arrow["\simeq", from=1-1, to=2-1]
    \arrow[from=1-2, to=1-3]
    \arrow["\simeq"', from=1-3, to=2-3]
    \arrow[from=2-1, to=2-3]  
  \end{cd}
  This implies there's $\si \in \GAL(\bQ(\mu_n) / \bQ)$
  which gives $x \mapsto x^p$ in $\AUT_{\GRP} \mu_n$.
\end{proof}

We can apply this to lines and circle constructions
to determine which regular $n$-gons are constructible
using lines and circles.

\begin{dfn}
  
  Let $K$ be a field and $x , y \in L / K$ some extension.
  We say $(x , y)$ is \emph{constructible in one step from $K$}
  when it they can be obtained as solutions to
  the following three kinds of simultaneous equations.
  \begin{enumerate}
    \item (Line-Line intersection) $a_1 x + b_1 y + c_1 = 0$ and 
    $a_2 x + b_2 y + c_2 = 0$.
    \item (Line-circle intersection)
    $a x + b y + c = 0$ and $x^2 + y^2 + A x + B y + C = 0$.
    \item (Circle-circle intersection)
    $x^2 + y^2 + A x + B y + C = 0$ and 
    $x^2 + y^2 + R x + S y + T = 0$
  \end{enumerate}
  We say $(x , y)$ is constructible from $K$
  when there exists a sequence $(x_0 , y_0) , \dots , (x_n , y_n)$
  with $x_0 , y_0 \in K$ and $x_n = x , y_n = y$ such that
  $(x_{i+1} , y_{i+1})$ is constructible from $K_i$ in one step
  and $K_{i+1} := K_i(x_{i+1} , y_{i+1})$.
\end{dfn}
\begin{prop}
  Let $K$ be a field and $x , y \in L / K$ some extension.
  Then $(x , y)$ is {constructible in one step from $K$}
  iff $[K(x , y) : K] \leq 2$.
  Hence $(x , y)$ is constructible from $K$ iff
  $[K(x , y) : K]$ is a power of 2.
\end{prop}
\begin{proof}
  The three kinds of simultaneous equations
  are all equivalent to solving at most a quadratic equation.
\end{proof}

\begin{eg}[Constructing regular pentagon]

  We ask if we can construct the regular pentagon from $\bQ$.
  This is equivalent to $(\cos(2\pi / 5) , \sin(2\pi/5))$
  being constructible from $\bQ$,
  which is equivalent to 
  $[\bQ(\cos(2\pi / 5) , \sin(2\pi/5)) : \bQ] = 2^n$.
  By chucking in $i$,
  this is equivalent to 
  $[\bQ(\cos(2\pi / 5) , \sin(2\pi/5), i) : \bQ] = 2^{n+1}$.
  Now $\bQ(\mu_5) \subs \bQ(\cos(2\pi / 5) , \sin(2\pi/5), i)$.
  If you chuck $i$ into $\bQ(\mu_5)$ then
  you can make $\sin(2 \pi / 5)$ because $\zeta := e^{2\pi i / 5} = 
  \cos(2 \pi / 5) + i \sin(2 \pi / 5)$.
  So the condition is equivalent to 
  $[\bQ(\mu_5) : \bQ] = 2^n$.
  By the theory of cyclotomic extensions over $\bQ$,
  $[\bQ(\mu_5) : \bQ] = \phi(5) = 4$
  so the regular pentagon is constructible.
  In fact we can derive a contruction by
  writing $\bQ \to \bQ(\mu_5)$ as a tower of degree 2 extensions
  solving quadratics we can construct.
  In this case, we are lucky because 
  $\cos(2 \pi / 5) = (\zeta + \zeta^{-1})/2$ which is fixed by
  the unique order two element in $\GAL(\bQ(\mu_5) / \bQ)$.
  Indeed, \[
    (\zeta + \zeta^{-1})^2 = \zeta^2 + \zeta^{-2} + 2
    = \zeta - \zeta^{-1} +1
  \]
  so \[
    \cos(2\pi / 5) = \frac{-1 + \sqrt{5}}{4}
  \]


\end{eg}

\begin{prop}[Characterization of constructible regular polygons]
  
  \textbf{TODO}
\end{prop}

\begin{proof}[Proof of Frobenius lifts.]

  Just as $K^G = \bQ$, we prove $A^G = \bZ$.
  The following is an ``integral version'' of
  the characterisation of finite simple extensions.
  \begin{lem}[Characterisation of integral elements]
    Let $K / \bQ$ be an extension and $\al \in K$.
    Then $\al$ is the root of a \emph{monic} $f \in \bZ[X]$
    iff $\bZ[\al]$ is finitely generated as a $\bZ$-module.
    If any of the above are satisfied, 
    we say $\al$ is \emph{integral over $\bZ$}.

    Consequently, 
    if $\al , \be \in K$ are integral over $\bZ$,
    then so is any $x \in \bZ[\al , \be]$.
    \begin{proof1}
      $(\implies)$ Same strategy as for algebraic elements.
      $(\limplies)$ Since $\bZ[\al] \subs K$ which does not have any
      torsion, it follows from Smith normal form that
      $\bZ[\al]$ has a $\bZ$-basis $x_1 , \dots , x_d$.
      We can write $x_i = g_i(\al)$ for some $g_i \neq 0 \in \bZ[X]$.
      Let $n =$ maximum of the degrees of $g_i$.
      Then $\al^{n+1} = \la_1 x_1 + \cdots + \la_d x_d$ for
      some $\la_i \in \bZ$.
      Expanding out $x_i = g_i(\al)$ implies
      $\al$ satisfies $X^{n+1} - \la_1 g_i(X) - \cdots - \la_d g_d(X)$,
      which is monic and in $\bZ[X]$.

      For the consequence,
      $\al , \be$ integral implies $\bZ[\al , \be]$ is finitely generated
      as a $\bZ$-module.
      This implies the submodule $\bZ[x] \subs \bZ[\al , \be]$ is also,
      and hence $x$ is integral over $\bZ$.
    \end{proof1}
  \end{lem}
  The lemma implies that every element in $A$ is integral over $\bZ$.
  Then for any $x \in A^G = A \cap K^G = A \cap \bQ$,
  we can write $x = a / b$ for $a , b \in \bZ$ coprime and $b \neq 0$,
  and $x^d + \la_1 x^{d - 1} + \cdots + \la_d = 0$ for some $\la_i \in \bZ$
  and $d > 0$.
  Then $a^d + \la_1 b a^{d - 1} + \cdots + \la_d b^d = 0$
  so $b$ divides $a$ and hence $b = \pm 1$.
  Thus $A^G = \bZ$.

  (1) 
  It is a consequence of Zorn's lemma that
  if $p \in A$ is not a unit, then such $\f{q}$ exists.
  We have $1 / p \notin A$ because
  if it were then $1 / p \in A^G = \bZ$.

  (2) (a) It is a field because $\f{q}$ is maximal
  and it is the splitting field of $\bar{f}$ because
  it is generated by the image of $S$,
  which gives all the roots of $\bar{f}$.
  Note that $\f{q} \cap \bZ$ is a proper ideal containing $(p)$
  so $\f{q} \cap \bZ = (p)$.

  (b) Write $f(T) = \prod_{a \in S}(T - a) \in \bZ[S][T]$.
  Then $\bar{f}(T) = \prod_{a \in S}(T - \bar{a}) \in \ka(\f{q})[T]$
  where $\bar{a} = a$ mod $\f{q}$.
  By assumption of $\bar{f}$ being separable,
  $a \neq b$ in $S$ implies $\bar{a} \neq \bar{b}$.
  So $S \to \bar{S}$ is an injection of finite sets and hence a bijection.
  The compatibility of actions is clear.
  
  (c) Injectivity follows from $S \to \bar{S}$ being bijective.
  Surjectivity is more subtle,
  we closely follow the argument from 
  \cite[\href{https://stacks.math.columbia.edu/tag/0BRJ}{Lemma 0BRJ}]{stacks-project}.
  \begin{lem}
    For primes $\f{q} , \f{q}_1 \subs A$ lying above
    $p$, there exists $\si \in G := \GAL(K / \bQ)$ such that
    $\si(\f{q}) = \f{q}_1$.
    It follows that there are finitely many primes
    lying above $p$.
    \begin{proof1}
      Suppose for a contradiction that
      there exists $x \in \f{q}_1$ and not in $\si(\f{q})$ for
      any $\si \in G$.
      Then $\prod_{\si \in G} \si(x) \in \bigcap_{\si \in G} \si(\f{q}_1)$.
      The latter is a proper ideal of $A^G = \bZ$ containing $(p)$
      so must be equal to $(p)$.
      But then we have $\prod_{\si \in G} \si(x) \in (p) \subs \f{q}$
      and hence $x \in \si^{-1}(\f{q})$ for some $\si \in G$,
      a contradiction.
    \end{proof1}
  \end{lem}
  By the Chinese remainder theorem, we have a surjection
  \[
    A \map{}{}
    \prod_{\f{q}_1 \in \ORB(\f{q})} A / \f{q}_1
  \]
  We know $\ka(\f{q})$ is a finite extension of $\bF_p$.
  So by the theory of finite fields and cyclotomic extensions,
  there exists $\bar{a}$ generating $\ka(\f{q})^\times$.
  Choose $a \in A$ such that $a = \bar{a}$ mod $\f{q}$
  and $0$ mod $\f{q}_1 \neq \f{q}$.
  Now we use the Galois theory trick : 
  $\min(a , \bQ)(T) = \prod_{\al \in \ORB(a)} (T - \al) \in K[T]$.
  Since $a \in A$ it follows that $\min(a , \bQ) \in A[T]$.
  Reducing $\f{q}$, we find that the minimal polynomial
  of $\bar{a}$ must divide $\prod_{\al \in \ORB(a)} (T - \bar{\al})$.
  This implies $\FROB(\bar{a}) = \bar{\si(a)}$ for some 
  $\si \in \GAL(K / \bQ)$.
  It STS $\si \in D(\f{q} / p)$.
  Let $x \in \f{q}$.
  Then $a x = 0$ mod $\bigcap_{\f{q}_1 \in \ORB(\f{q})} \f{q}_1$
  so $\si(a x) = 0$ mod $\bigcap_{\f{q}_1 \in \ORB(\f{q})} \f{q}_1$.
  Looking at the $A / \f{q}$ component,
  $0 = \si(x) \si(a) = \si(x) \bar{a}^p$ mod $\f{q}$ 
  which implies $\si(x) = 0$ mod $\f{q}$.
\end{proof}

\section{Bonus : Sneak peak at $p$-adic and perfectoid fields}

\textbf{To be fleshed out : }
\begin{enumerate}
  \item $\bF_p[T]$
  \item $\bF_p[[T]] := \LIM_{n \geq 0} \bF_p[T] / (T^{n+1})$.
  % If $T f = 0$ then for all $n \geq 0$, we have 
  % $T(f_{n , 0} + \cdots + f_{n , n}T^n) = 0$ mod $T^{n+1}$
  % which implies $f_{n , 0 \leq i < n} = 0$.
  % By compatibility, we have $f_{n , n} = f_{n+1 , n} = 0$.
  % So $f = 0$ mod $T^{n+1}$ for all $n \geq 0$, i.e. $f = 0$.
  As a set we have \[
    \prod_{n \geq 0} \bF_p \map{\sim}{} \bF_p[[T]]
  \]
  by $(a_n) \mapsto (\sum_{0 \leq d \leq n} a_d T^d \text{ mod } T^{n+1})_n$.
  One can check that the coefficients of
  sums and products are as one would expect for power series.
  We have an induced ring morphism \[
    \bF_p[T] \to \bF_p[[T]]
  \]
  It is injective because if one has a polynomial $f \in \bF_p[T]$
  which is divisible by $T^{n+1}$ for all $n \geq 0$ then $f = 0$.
  Notice in the $T$-adic completion,
  $1 - T$ has an inverse given by $\sum_{d \geq 0} T^d$.
  Indeed, mod $T^{n+1}$ we have \[
    (1 - T)(1 + T + \cdots + T^{n}) = 1
  \]
  In fact, $(T)$ is the unique maximal ideal of $\bF_p[[T]]$.
  (Hint : One idea is to use geometric series.)
  The geometric intuition is that 
  $\bF_p[T]$ is the ring of functions on the affine line,
  $\bF_p[T] / (T)$ is the ring of functions 
  at the point $\set{0}$ in the affine line.
  $\bF_p[T] / (T^2)$ is the ring of functions on 
  a subspace of the affine line that's a tiny bit bigger than just $\set{0}$.
  It's not bigger to the point that $T$ can take on any values
  other than $0$ because $T^2 = 0$.
  But this space is large enough to know about
  the first derivative of functions at $\set{0}$.
  This is call the \emph{first order infinitesimal neighbourhood of $\set{0}$}.
  Similarly for $\bF_p[T] / (T^{n+1})$.
  Finally, $\bF_p[[T]]$ is the ring of functions
  on the union of these infinitesimal neighbourhoods.
  The fact that anything outside $(T)$ is invertible
  says anything that is non-zero at zero is invertible.
  In this sense, we still don't have points other than zero.

  One can show $\bF_p[[T]]$ is a domain.
  Let $f , g \in \bF_p[[T]]$ with $f g = 0$.
  Then $f_0 g_0 = 0$ so WLOG $f_1 \in \bF_p^\times$ and $g_0 = 0$.
  Then $f_0 g_1 + f_1 g_0 = 0$ implies $g_1 = 0$.
  By induction, $g_n = 0$ for all $n \geq 0$
  so $g = 0$.
  This means we can take fraction field \[
    \bF_p((T)) := \FRAC\,\bF_p[[T]] \simeq \bF_p[[T]][1 / T]
  \]
  which intuitively is the ring of functions on the
  \emph{punctured disk around zero}.

  \item $\bZ$ is similar to $\bF_p[T]$ in the sense that
  elements can be written uniquely as 
  polynomials with coefficients in $\set{0 , \dots , p - 1}$.
  In $\bZ$ the ``variable'' is $p$ and unlike $\bF_p$,
  there's a non-trivial ``carrying over'' of coefficients
  when adding elements. For example in $\bZ$
  \[
    (0 \cdot 1 + 3 \cdot 5) + 
    (0 \cdot 1 + 2 \cdot 5) = 
    (0 \cdot 1 + 0 \cdot 5 + 1 \cdot 5^2)
  \]
  whilst in $\bF_5[T]$
  \[
    (0 + 3 T) +
    (0 + 2 T)
    = 0
  \]
  An idea is that we can play the same game of completion
  with $p$ instead of $T$.
  The resulting ring is called the \emph{$p$-adic integers}.
  \[
    \bZ_p := \LIM_{n \geq 0} \bZ / (p^{n+1})
  \]
  By the same argument as for $\bF_p[T]$ we have an injection
  \[
    \bZ \to \bZ_p
  \]
  Again, as sets we have \[
    \prod_{n \geq 0} \set{0 , \cdots , p - 1}
    \map{\sim}{} \bZ_p
  \]
  \[
    (a_n)_n \mapsto (\sum_{0 \leq d \leq n} a_d p^d \text{ mod }p^{n+1})_n
  \]
  This is not too useful because of the non-trivial ``carrying over''
  of coefficients when adding and multiplying unlike the case of $\bF_p[[T]]$.

  One can also show $\bZ_p$ has $(p)$ as the unique maximal ideal.
  The key is that in$\bZ / (p^{n+1})$,
  $p$ is nilpotent so any maximal ideal must contain $(p)$,
  which is maximal itself.
  It follows that $(p)$ is the unique maximal ideal of $\bZ / (p^{n+1})$
  for all $n \geq 0$.
  So if $x \in \bZ_p$ with $x \neq 0$ mod $p$,
  then $x$ mod $p^{n+1}$ is invertible for all $n \geq 0$
  and hence $x$ is invertible.
  \footnote{
    This strategy also works for $\bF_p[[T]]$.
  }

  Geometrically, if one pretends $\bZ$ is the ring of functions on some space
  and $p$ is a point in that space,
  then one can imagine $\bF_p = \bZ / (p)$ as
  the ring of functions at the point $p$ and
  $\bZ_p$ as the ring of functions on the formal disk around $p$.

  One can also show $\bZ_p$ is a domain.
  The key is $p$-torsion-free.
  \begin{lem}
    For $x \in \bZ_p$, if $p x = 0$ then $x = 0$.

    \begin{proof1}
      Write $x = \sum_{n \geq 0} x_n p^n$ with 
      $x_n \in \set{0 , \cdots , p - 1} \subs \bZ$.
      Mod $p^2$ we have $p x_0 = 0$ so $x_0 = 0$.
      Mod $p^3$ we have $0 = p(x_1 p + x_2 p^2) = x_1 p^2$ so $x_1 = 0$.
      By induction $x_n = 0$ for all $n$.
    \end{proof1}
  \end{lem}
  Let $x , y \in \bZ_p$ with $x y = 0$.
  Writing both as $p$-adic expansions with coefficients in $[0 , p-1] \cap \bZ$
  gives $x_0 y_0 = 0$ mod $p$ and hence $x_0 = 0$ or $y_0 = 0$.
  WLOG $x_0 \in \bF_p^\times$ and $y_0 = 0$.
  Then we can write $0 = x y = p x (y_1 + y_2 p + \cdots)$
  and so $x (y_1 + y_2 p + \cdots) = 0$ in $\bZ_p$.
  By induction, $y_n = 0$ for all $n$ and hence $y = 0$.

  Thus, we can take fraction fields and obtain
  the \emph{$p$-adic rationals}.
  \[
    \bQ_p := \FRAC\,{\bZ_p} \simeq \bZ_p[1 / p]
  \]
  Intuitively, this is the ring of functions on the punctured disk around $p$.

  \item Why is this useful? 
  Recall in the section on cyclotomic extensions,
  we proved that $\bF_p^\times$ is cyclic effectively by counting.
  Let us give a different proof by relating $\bF_p^\times$
  with $(p-1)$-th roots of unity in $\bC$, which we know is cyclic
  because it is generated by $e^{2 \pi i / (p - 1)}$.

  Recall we wrote elements of $\bZ_p$ as 
  power series in $p$ with coefficients by picking
  $\set{0 , \dots , p-1} \subs \bZ \subs \bZ_p$ as lifts of $\bF_p^\times$
  under $\bZ_p \to \bF_p$.
  This does not reflect the additive nor multiplicative
  nature of elements in $\bF_p^\times$.
  We cannot expect addition of lifts to be respected because $p = 0$ in $\bF_p$
  but $p \neq 0$ in $\bZ \subs \bZ_p$.
  We now show that there is a better choice of coefficients
  which is multiplicative.
  More precisely,
  \begin{lem}
    There exists a unique multiplicative map $[\_] : \bF_p \to \bZ_p$
    such that $[x] = x$ mod $p$.
   
    \begin{proof1}
      The idea is that for $x = y \in \bF_p = \bZ / (p)$,
      although we cannot in general lift $x , y$ so that
      $x = y$ mod $p^2$ we do have $x^p = y^p$ mod $p^2$ by binomial expansion.
      Take $x \in \bF_p$,
      we define $[x]$ mod $p^{n+1}$ for each $n \geq 0$ : 
      \begin{enumerate}
        \item Take $x^{1 / p} \in \bF_p$,
        which is unique by the Frobenius being bijective.

        In general, take $x^{1 / p^n}$.
        \item Take \emph{any} lift $x_1 \in \bZ / (p^2)$ of $x^{1 / p}$.
        
        For general $n$ take any lift $x_n \in \bZ / (p^{n+1})$
        of $x^{1 / p^n}$.
        \item Take $[x]_1 := x_1^p \in \bZ / (p^2)$.
        This only depends on the value of $[x]_1$ mod $p$
        which is $[x]_1 = x_1^p = (x^{1 / p})^p = x$ mod $p$.

        For any $n$, take $[x]_n := x_n^{p^n} \in \bZ / (p^n)$.
        This depends only on the value of $[x]_n$ mod $p$
        which is again $x$.
        \item Because $(x^{1 / p^{n+1}})^p = x^{1 / p^n}$ mod $p$
        we have $[x]_{n+1} = (x_{n+1}^p)^{p^n} = x_n^{p^n} = [x]_n$ mod $p^n$.
        This defines $[x] \in \bZ_p$.
      \end{enumerate}
      For $x , y \in \bF_p$,
      $x_n y_n$ lifts $(x y)^{1 / p^n}$ so 
      $[x]_n [y]_n = x_n^{p^n} y_n^{p^n} = [xy]_n$ mod $p^{n+1}$.
      It follows that $[x][y] = [x y]$.
      Uniqueness is straightforward to check.
    \end{proof1}
  \end{lem}

  In particular, $[\bF_p^\times] \subs \bZ_p$ gives
  $(p-1)$-th roots of unity!
  Furthermore, mod $p$ gives a group isomorphism 
  $[\bF_p^\times] \simeq \bF_p^\times$.
  So we have lifting the $(p-1)$-th roots of unity in characteristic $p$
  to characteristic zero.
  
  Now note $\bZ \subs \bZ_p$ induces a field extension 
  $\bQ \to \bQ_p$ (of infinite degree).
  Take $\bQ([\bF_p^\times])$ which is a finite subextension.
  This must be a splitting field for $X^{p-1} - 1 \in \bQ[X]$.
  By the embedding theorem,
  we must have some isomorphism
  $\bQ([\bF_p^\times]) \simeq \bQ(e^{2 \pi i / (p-1)})$
  as extensions of $\bQ$ since both are splitting fields of $X^{p-1} - 1$.
  Under this unknown isomorphism,
  $[\bF_p^\times] \simeq \<e^{2\pi i / (p-1)}\>$.
  Thus we conclude $\bF_p^\times$ is cyclic.

  \item Define $\bF_p((t^{1 / p^\infty}))$ and $\bQ_p^\infty$.
  State Fontaine--Winterberger isomorphism.
  \cite{FW83}

  These are examples of \emph{perfectoid fields}.
  Just as fields generalise to rings,
  perfectoid fields generalise to \emph{perfectoid rings}.
  The applications of these rings falls in
  the area called \emph{$p$-adic Hodge theory}.
  Describing this is beyond the scope of these talks.
  Let's just say they were significant enough to win Peter Scholze
  the Fields medal in 2018!

  \item (If time permits) 
  Tilting $\bQ_p^\infty$ to $\bF_p((t^{1 / p^\infty}))$.

\end{enumerate}



\printbibliography

\end{document}